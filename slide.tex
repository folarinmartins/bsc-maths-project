\documentclass[10pt]{beamer}
% \mode<presentation>{\usetheme{Warsaw}}%Boadilla Antibes Singapore Hannover Madrid
\usetheme{Warsaw}
\usepackage{amsmath,amssymb,latexsym,enumerate}
\usepackage{longtable}
\title{Solution to Attribute-Based Assignment Problems}
\subtitle{A Primer to the University Timetabling Problem}
\author{\textbf{Folarin, Wasiu Jr.}\\ MTH/2014/014}
\institute{\textbf{Dr. B. S. Ogundare}\\ \textit{Supervisor}}
\date{\today}
\logo{\includegraphics[height=0.5cm]{figures/oau.png}}
\begin{document}
	\begin{frame}
		\titlepage
	\end{frame}

	% \begin{frame}
	% 	\frametitle{Outline}
	% 	\tableofcontents
	% \end{frame}

	\section{Introduction}
		\subsection{Core Concepts}
			\begin{frame}
				\frametitle{Introduction}
				\begin{block}{Claim}
					Given any set of objects $U = \{u_1, u_2,...,u_n\}$ each possessing certain attributes, and a set of constraining conditions for assignment, we can:
					\begin{enumerate}
						\item Construct functions describing those constraints
						\item Assign each object to a class of best fit
					\end{enumerate}
				\end{block}
			\end{frame}
			\begin{frame}
				\frametitle{The Nature of Attributes}
				\begin{definition}
					An Attribute is any characteristic of a group of objects that can serve as a basis for classification
				\end{definition}
				\begin{example}
					\begin{itemize}
						\item Age
						\item State of Origin
						\item Courses offered in current Semester
						\item Body mass
						\item O'Level Results
						\item etc...
					\end{itemize}
				\end{example}
			\end{frame}

			\begin{frame}
				\frametitle{Partitions}
				\begin{definition}
					A Partition, is a classification of objects based on an \underline{ideal value} of one or more \texttt{attributes}
				\end{definition}
				\begin{example}
					\begin{itemize}
						\item Teens, Young Adults,... (based on the Age attribute of objects)
						\item Hausa, Igbo, Yoruba,... (based on the State of Origin attribute)
						\item etc...
					\end{itemize}
				\end{example}
			\end{frame}

			\begin{frame}
				\frametitle{Constraints}
				\vfill
				\begin{definition}<1->
					A Constraint function, $\mu$, matches the \underline{Ideal value} of a partition, $p_i$, to the \texttt{attribute} value of an object, $u_j$, and assigns a membership value viz:
					\begin{align}
						\mu(p_i,u_j) &\mapsto [0,1] \textit{\ \ \  i,j $\in \mathbf{Z}$ }\notag
					\end{align}

				\end{definition}
				\vfill
				\begin{example}<2->
					\begin{description}
						\item [NYSC:] No student should be posted to their State of Origin.
						\item [ANALYSIS:] Students should be posted as far as possible from their States of Origin
						\item [ATTRIBUTE:] The distance of each state from a common point
					\end{description}
				\end{example}
			\end{frame}

			\begin{frame}
				\frametitle{Classes of Constraints}
				\vfill
				\begin{definition}<1->
					\begin{itemize}
						\item \textbf{Binary Constraints} assign a membership value $v \in \{0,1\}$ to each object
						\item \textbf{Continuous Constraints} assign a membership value $v \in [0,1]$ to each object
					\end{itemize}
				\end{definition}
				\begin{definition}<1->
					\begin{itemize}
						\item \textbf{Inclusion Constraints} admit objects based on proximity to the \texttt{ideal value}
						\item \textbf{Exclusion Constraints} admit objects based on distance from the \texttt{ideal value}
					\end{itemize}
				\end{definition}
			\end{frame}

	\section{A Sample Problem: NYSC Posting}
		\subsection{Problem Statement}
			\begin{frame}
				\frametitle{The NYSC Posting Problem}
				\begin{block}{Problem Statement}
					Given a number of graduates, we seek to post them to Service States under the following constraints:
					\begin{enumerate}
						\item No students should be posted to their State of Origin
						\item No students should be posted to their States of study
						\item The maximum capacity of each state must not be exceeded%\footnote{This algorithm innately caters to capacity constraints}
						\item All students should be assigned
					\end{enumerate}
				\end{block}		
			\end{frame}

			\begin{frame}
				\frametitle{The States - Partitions}
				\begin{center}
					\begin{tabular}{|c|c|c|c|c|}
						\hline
						\textbf{Serial} & \textbf{State} & \textbf{Capacity} \\
						\hline
						1 & Osun State & 3\\
						\hline
						2 & Kwara State & 2\\
						\hline
						3 & Enugu State & 1\\
						\hline
						4 & Akwa-Ibom State & 5\\
						\hline
						5 & Kano State & 1\\
						\hline
					\end{tabular}		
				\end{center}
			\end{frame}
			\begin{frame}
				\frametitle{The Graduates - Objects}				
				\begin{center}
					\begin{tabular}{|c|c|c|c|}
						\hline
						\textbf{Serial} & \textbf{Name} & \textbf{State of Origin} & \textbf{University}\\
						\hline
						$u_{1}$ & GR01 & Kwara(2) & OAU(1)\\
						\hline
						$u_{2}$ & GR02 & Kwara(2) & BUK(5)\\
						\hline
						$u_{3}$ & GR03 & Kwara(2) & NSUKKA(3)\\
						\hline
						$u_{4}$ & GR04 & Kwara(2) & UNILORIN(2)\\
						\hline
						$u_{5}$ & GR05 & Akwa-Ibom(4) & OAU(1)\\
						\hline
						$u_{6}$ & GR06 & Akwa-Ibom(4) & BUK(5)\\
						\hline
						$u_{7}$ & GR07 & Akwa-Ibom(4) & NSUKKA(3)\\
						\hline
						$\vdots $ & $\vdots $ & $\vdots $ & $\vdots $ \\
						\hline
						$u_{16}$ & GR16 & Kano(5) & AKSU(4)\\
						\hline
						$\vdots $ & $\vdots $ & $\vdots $ & $\vdots $ \\
						\hline
						$u_{20}$ & GR20 & Enugu(3) & AKSU(4)\\
						\hline
					\end{tabular}		
				\end{center}
			\end{frame}
			\subsection{The Constraints}
			\begin{frame}
				\frametitle{Constraint A, $\mu_A\colon u\mapsto [0,1]$}
				Constraint A, $\mu_A$: No student should be posted to their State of Origin
				\begin{block}{Illustration}
					\begin{equation}
						\mu_{A_j}(u_i) = \frac{\left(j-s_A(u_i)\right)^2}{1 + \left(j-s_A(u_i)\right)^2} \notag
					\end{equation}
				\end{block}
				\begin{example}
					\begin{align}
						\mu_{A_1}(u_1) &= \frac{\left(1-s_A(u_1)\right)^2}{1 + \left(1-s_A(u_1)\right)^2} \notag\\
						&= \frac{\left(1-2\right)^2}{1 + \left(1-2\right)^2} \notag \\
						&= 0.5 \notag \\
						&\vdots \notag
					\end{align}
				\end{example}
			\end{frame}
			\begin{frame}
				\frametitle{Constraint B, $\mu_B\colon u\mapsto [0,1]$}
				Constraint B, $\mu_B$: No student should be posted to their State of studies
				\begin{block}{Illustration}
					\begin{equation}
						\mu_{B_j}(u_i) = \frac{\left(j-s_B(u_i)\right)^2}{1 + \left(j-s_B(u_i)\right)^2} \notag
					\end{equation}
				\end{block}
				\begin{example}
					\begin{align}
						\mu_{B_1}(u_1) &= \frac{\left(1-s_B(u_1)\right)^2}{1 + \left(1-s_B(u_1)\right)^2}\notag\\
						&= \frac{\left(1-1\right)^2}{1 + \left(1-1\right)^2} \notag \\
						&= 0 \notag \\
						&\vdots \notag
					\end{align}
				\end{example}
			\end{frame}

			\subsection{Assignment Schedule}
			\begin{frame}
				\frametitle{The Solution}				
				\begin{center}
					\begin{tabular}{|c|c|c|c|c|}
						\hline
							\textbf{Name} & \textbf{S. of Origin} & \textbf{University} & \textbf{Posting} & \textbf{Membership} \\
							\hline
							GR06 & Akwa-Ibom & BUK, Kano & Osun & 85\% \\
							\hline
							GR16 & Kano & AKSU, Akwa-Ibom & Osun & 84\% \\
							\hline
							GR04 & Kwara & UNILORIN, Kwara & Akwa-Ibom & 64\% \\
							\hline
							GR01 & Kwara & OAU, Osun & Akwa-Ibom & 45\% \\
							\hline
							GR03 & Kwara & NSUKKA, Enugu & Kano & 88\% \\
							\hline
							GR05 & Akwa-Ibom & OAU, Osun & Kano & 47\% \\
							\hline
							GR07 & Akwa-Ibom & NSUKKA, Enugu & Kano & 45\% \\
							\hline
							GR20 & Enugu & AKSU, Akwa-Ibom & Kano & 40\% \\
							\hline
							$\vdots$ & $\vdots$ & $\vdots$ & $\vdots$ & $\vdots$ \\
							\hline
					\end{tabular}		
				\end{center}
			\end{frame}
			
			\section{End}
			\begin{frame}
				\frametitle{The Presentation}	
				\begin{center}			
					\begin{quote}
						\centering \textbf{Thank You!}
					\end{quote}					
				\end{center}
			\end{frame}


		% \begin{frame}
		% 	\frametitle{List}
		% 	\begin{itemize}
		% 		\item Point A
		% 		\item Point B
		% 		\begin{itemize}
		% 			\item part 1
		% 			\item part 2
		% 		\end{itemize}
		% 		\item Point C
		% 		\item Point D
		% 	\end{itemize}
		% \end{frame}

		% \begin{frame}
		% 	\frametitle{Using Columns}
		% 	\begin{columns}
		% 		\column{0.5\textwidth}
		% 			<text>
		% 		\column{0.5\textwidth}
		% 			\centering
		% 			\begin{figure}[!htb]
		% 				\center{\includegraphics[width=\textwidth]
		% 				{figures/fig1.png}}
		% 			\caption{\label{fig:fig1} My figure.  An example of a cool figure}
		% 		\end{figure}
		% 	\end{columns}
		% \end{frame}

		% \begin{frame}
		% 	\frametitle{Listing}
		% 	\begin{description}
		% 		\label{terms}
		% 		\item[API] Application Programming Interface
		% 		\item[LAN] Local Area Network
		% 		\item[ASCII] American Standard Code for Information Interchange
		% 	\end{description}
		% 	\begin{block}{Block Title}
		% 		Lorem ipsum dolor sit amet, consectetur adipisicing elit,
		% 		sed do eiusmod tempor incididunt ut labore et
		% 		dolore magna aliqua.
		% 	\end{block}
		% \end{frame}
		% \begin{frame}
		% 	\begin{definition}
		% 		A prime number is a number that...
		% 	\end{definition}
		% 	\begin{example}
		% 		A prime number is a number that...
		% 	\end{example}
		% 	\begin{theorem}[Pythagoras]
		% 		$ a^2 + b^2 = c^2$
		% 	\end{theorem}
		% 	\begin{corollary}
		% 		$ x + y = y + x  $
		% 	\end{corollary}
		% 	\begin{proof}
		% 		$\omega +\phi = \epsilon $
		% 	\end{proof}
		% 	\hyperlink{terms}{\beamergotobutton{go to terms page}}
		% \end{frame}

		% \begin{frame}
			% \frametitle{List}
			% \begin{itemize}
			% \pause
			% \item Point A
			% \pause
			% \item Point B
			% \begin{itemize}
			% \pause
			% \item part 1
			% \pause
			% \item part 2
			% \end{itemize}
			% \pause
			% \item Point C
			% \pause
			% \item Point D
			% \end{itemize}
		% \end{frame}
		% \begin{frame}
		% 	\frametitle{More Lists}
		% 	\begin{enumerate}[(I)]
		% 	\item<-2,4-5,7> Point A
		% 	\item<2-> Point B
		% 	\begin{itemize}
		% 	\item<3-> part 1
		% 	\item<4-> part 2
		% 	\end{itemize}
		% 	\item<5-> Point C
		% 	\item<6-> Point D
		% 	\item<7> Point D
		% 	\end{enumerate}
		% 	\textbf<2>{Example Text}
		% 	\textit<2>{Example Text}
		% 	\textsl<2>{Example Text}
		% 	\textrm<2>{Example Text}
		% 	\textsf<2>{Example Text}
		% 	\textcolor<2>{orange}{Example Text}
		% 	\alert<2>{Example Text}
		% 	\structure<2>{Example Text}
		% \end{frame}
\end{document}
