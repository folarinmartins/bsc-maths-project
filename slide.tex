\documentclass[10pt]{beamer}
% \mode<presentation>{\usetheme{Warsaw}}%Boadilla Antibes Singapore Hannover Madrid
\usetheme{Warsaw}
\title{Solution to Attribute-Based Assignment Problems}
\subtitle{A Primer to the University Timetabling Problem}
\author{Folarin, Wasiu Jr.}
\institute{Obafemi Awolowo University\\Department of Mathematics}
\date{\today}
\logo{\includegraphics[height=0.5cm]{figures/oau.png}}

\begin{document}
	\begin{frame}
		\titlepage
	\end{frame}
	
	\begin{frame}
		\frametitle{Outline}
		\tableofcontents
	\end{frame}

	\section{Section 1}
		\subsection{sub a}
		
		\begin{frame}
			\frametitle{Title}
			Lorem ipsum dolor sit amet, consectetur adipisicing elit, sed do eiusmod tempor incididunt ut labore et dolore magna aliqua.
		\end{frame}
		
		\begin{frame}
			\frametitle{List}
			\begin{itemize}
				\item Point A
				\item Point B
				\begin{itemize}
					\item part 1
					\item part 2
				\end{itemize}
				\item Point C
				\item Point D
			\end{itemize}
		\end{frame}
		
		\begin{frame}
			\frametitle{Using Columns}
			\begin{columns}
				\column{0.5\textwidth}
					<text>
				\column{0.5\textwidth}
					\centering
					\begin{figure}[!htb]
						\center{\includegraphics[width=\textwidth]
						{figures/fig1.png}}
					\caption{\label{fig:fig1} My figure.  An example of a cool figure}
				\end{figure}
			\end{columns}
		\end{frame}
		
		\begin{frame}
			\frametitle{Listing}
			\begin{description}
				\label{terms}
				\item[API] Application Programming Interface
				\item[LAN] Local Area Network
				\item[ASCII] American Standard Code for Information Interchange
			\end{description}
			\begin{block}{Block Title}
				Lorem ipsum dolor sit amet, consectetur adipisicing elit, 
				sed do eiusmod tempor incididunt ut labore et 
				dolore magna aliqua.
			\end{block}
		\end{frame}
		\begin{frame}
			\begin{definition}
				A prime number is a number that...
			\end{definition}
			\begin{example}
				A prime number is a number that...
			\end{example}
			\begin{theorem}[Pythagoras] 
				$ a^2 + b^2 = c^2$
			\end{theorem}
			\begin{corollary}
				$ x + y = y + x  $
			\end{corollary}
			\begin{proof}
				$\omega +\phi = \epsilon $
			\end{proof}
			\hyperlink{terms}{\beamergotobutton{go to terms page}}
		\end{frame}
		
		\begin{frame}
			% \frametitle{List}
			% \begin{itemize}
			% \pause
			% \item Point A
			% \pause
			% \item Point B
			% \begin{itemize}
			% \pause
			% \item part 1
			% \pause
			% \item part 2
			% \end{itemize}
			% \pause
			% \item Point C
			% \pause
			% \item Point D
			% \end{itemize}
		\end{frame}
		\begin{frame}
			\frametitle{More Lists}
			\begin{enumerate}[(I)]
			\item<-2,4-5,7> Point A
			\item<2-> Point B
			\begin{itemize}
			\item<3-> part 1
			\item<4-> part 2
			\end{itemize}
			\item<5-> Point C
			\item<6-> Point D
			\item<7> Point D
			\end{enumerate}
			\textbf<2>{Example Text}
			\textit<2>{Example Text}
			\textsl<2>{Example Text}
			\textrm<2>{Example Text}
			\textsf<2>{Example Text}
			\textcolor<2>{orange}{Example Text}
			\alert<2>{Example Text}
			\structure<2>{Example Text}
		\end{frame}
\end{document}
