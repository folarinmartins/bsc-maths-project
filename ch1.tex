% pdflatex ch1.tex
\documentclass[a4paper,openany]{book}
\usepackage{amsmath,amssymb,latexsym}
\usepackage[colorlinks=true,linkcolor=black,urlcolor=black,bookmarksopen=true]{hyperref}
\title{A SOLUTION TO ATTRIBUTE-BASED ASSIGNMENT PROBLEMS: A PRIMER SOLUTION TO THE UNIVERSITY TIMETABLING PROBLEM}
\date{September 2018}
\author{Folarin, Wasiu Junior}
\begin{document}
    \normalsize
	\pagenumbering{arabic}	
	\frontmatter
	\begin{titlepage}
		\begin{center}
			A SOLUTION TO ATTRIBUTE-BASED ASSIGNMENT PROBLEMS:\\
			A PRIMER TO THE UNIVERSITY TIMETABLING PROBLEM
		\end{center}
		\vfill
		\begin{center}
			BY
		\end{center}
		\vfill
		\begin{center}
			FOLARIN WASIU JUNIOR\\[0.25in]
			MTH/2014/014
		\end{center}
		\vfill
		\begin{center}
			A PROJECT SUBMITTED TO THE DEPARTMENT OF MATHEMATICS, FACULTY OF SCIENCE,\\
			OBAFEMI AWOLOWO UNIVERSITY, ILE-IFE, NIGERIA\\
			IN PARTIAL FULFILMENT OF THE REQUIREMENTS FOR THE AWARD OF 
			BACHELOR OF SCIENCE (B.Sc Hons) DEGREE IN MATHEMATICS\\ \bigskip
			FEBRUARY, 2019
		\end{center}
	\end{titlepage}
	\chapter[CERTIFICATION]{\centering CERTIFICATION}
	I certify that FOLARIN, Wasiu Junior has fulfilled all requirements for the award of B.Sc (Hons) Degree in Mathematics. This project is the result of the research work carried out by him under my supervision during the course of his undergraduate studies in the Department of Mathematics, Obafemi Awolowo University, Ile-Ife, Nigeria.
	\\[1in]
	\rule{1.5in}{0.1pt} \hfill \rule{1in}{0.4pt}
	\\[2ex]
	Dr. B. S. Ogundare \hfill Date \hspace{4ex}
	\\[1ex]
	\textit{Supervisor}
	\\[1.5in]
	\rule{1.5in}{0.1pt} \hfill \rule{1in}{0.4pt}
	\\[2ex]
	Dr. O. P. Layeni \hfill Date \hspace{4ex}
	\\[1ex]
	\textit{Head of Department}
	
	\chapter[DEDICATION]{ \centering DEDICATION}
		This work is dedicated to the Creatives among us and all those who have a deep and rather insatiable love for the world of logic, analytical reasoning and critical thinking...
		\begin{center}
			\textit{May all the good dreams come to fruition!}
		\end{center}
	\chapter[ACKNOWLEDGEMENT]{ \centering ACKNOWLEDGEMENT}
		Great acknowledgement to the Most High: The ONLY Wise One. And to my family, friends and colleagues for their much love. And specially to my Supervisor, Dr. B. S. Ogundare, for giving me the freedom and support to develop the techniques presented in this work.
	\chapter[ABSTRACT]{ \centering ABSTRACT}
		In this work, we take a mathematical exploration of certain assignment problems involving assigning a set of objects to non-overlapping subsets (partitions), in which assignment is based on certain properties possessed by the objects and the partitioning sets. Putting this in perspective, we attempt to solve a sample NYSC assigment problem. The algorithm is developed in this work, and certain notes on improvement cited.  We also attempt to generalize this technique to the more general and complex University Timetabling problem.
	% \listoffigures
	\tableofcontents
	\listoftables
	\mainmatter
    \linespread{2.0}
    \chapter{Introduction}
        \paragraph{In}
        introducing this study, we shall proceed under the following sub-headings:
        \begin{itemize}
            \item Preamble
            \item Motivation
            \item A Sample Problem
            \item Objective of the study
        \end{itemize}
        \section{Preamble}
        \paragraph{In}
			everyday human, corporate and industrial activities, we are faced with the challenges of optimal resource assignment. In the face of limited resources, low tolerance for scrap and rework, need for greater degree of accuracy and scalability, optimization of resources is paramount.
            Over the years, several Mathematical Optimization techniques in Operations Research (OR) specifically, have been developed to help with these class of problems witha good amount of success recorded. Nonetheless, there still remains a great room for improvement on these. Classical approaches to solving assignment problems, among others include the well revered Linear Programming techniques.
		\section{Motivation}
			\paragraph{The University Timetable Problem:}
			This study, though it doesn't directly proffer a solution to this problem, it attempts to solve a simplified variant of it.
			\\In Universities, there are usually a good number of students taking courses from a large pool, and in varying mixes, based on Faculty and Institutional requirements. Typically, there are limited facilities such as lecture halls to assign to each course. The problem therefore arises in creating an examination timetable that caters to all the courses, eliminating, or at least minimizing stress and ensuring usability of the timetable.
			\\Studies show that, in order to prepare a well usable timetabling system, provision has to be made for such considerations as the following parameters:
			\begin{enumerate}
				\item \underline{Time Clash}: A student group cannot have more than one exam at one time
				\item \underline{Semester Clash}: Student groups from the same major but in different semesters cannot have exams at the same time
				\item \underline{Core Papers}: A student group cannot take more than one core exam a day
				\item \underline{Daily Maximum}: A student group cannot have more than one exam per day
				\item \underline{Difficulty Level}: A student group cannot have two difficult exams in two consecutive days
				\item \underline{Capacity}: The total number of students seating for exams in a particular time cannot exceed predefined limits
				\item \underline{One-period Execution}: Each course must occur at the same time for all student groups
				\item \underline{Periodic Unavailability}: Some exams may not be scheduled at particular time slots
				\item \underline{Comprehensiveness of Coverage}: All exams on the timetable have to be assigned
				\item \underline{Pre-assignment Capabilities}: A course may be assigned to a particular target time slot
				\item \underline{Controllable Exams Conclusion}: Student in all student groups should conclude their exams at approximately the same time
			\end{enumerate}
			The foregoing attributes have been cited as some of the desirable properties of a credible timetabling system. Proffering credible solutions to the timetabling problem has been a major challenge in the wide, that annually, the International Timetabling Competition 2019 \cite{itc2019} holds to select a number of outstanding solutions to the timetabling problem.\\
			Although several aspects of this problem have been attempted over the years, it is still a long way from being efficiently solved to tackle the highlighted problems
            \paragraph{}
            	This study takes a whole new look at Mathematical Optimization: what if our optimization goal doesn't involve minimizing ( or maximizing ) any variable or group of variables such as cost, profit, or even price? What if we're on the otherhand interested in
                \begin{verse}
                    \textit{
                        \begin{center}
                        Allocation of objects to non-intersecting groups, based on attributes possessed by the objects, and one or more constraining conditions
                        \end{center}
                   }
                \end{verse}
                Mathematical Optimization techniques, have not been formulated to solve problems as this, and thus the motivaiton for this study.
        \section{A Reduction Problem}
            \paragraph{Consider the following problem statement:}
            In deploying intending NYSC members to a serving state, the following constraints need be satisfied:
            \begin{enumerate}
                \item \underline{State of Origin}: No student should be deployed to his/her state of origin
                \item \underline{State of Study}: No student should be deployed to his state of study
                \item \underline{Comprehensiveness of Coverage}: All enlisted students must be deployed
                \item \underline{Capacity}: No state should be allocated more students that it has resources to accommodate
            \end{enumerate}
            The foregoing problem statement is an attempt at reducing the more complex University Timetabling problem to a simpler form where the parameters involved are better appreciated. This study focuses on resolving the simpler NYSC batching problem, and makes recommendations on applying the concepts herein in tackling the more complex University Timetabling Problem.
		\section{Objective}
            \paragraph
                The aim of this study is to:
                \begin{itemize}
                    \item Develop a mathematical algorithm to solve assignment problems involving the distribution of candidate objects into non-overlapping groups, putting into considerations several constraining conditions
                    \item Create a simulation, exercising this model on a sample problem
                    \item Sketch a path to evolving this class of solution to more complex University Timetabling problem, owing to its more complex constraints formulation.
                \end{itemize}        
				Throughout this text, all mathematical notations retain their usual meanings, unless otherwise stated.
		\section{Definition of Terms}
            \subsection{Membership Probability or Degree}
               Each assignable object in all cases studied in this text, belong to each assignable slot with varying degrees of membership, usually a value in the interval [0,1]. An object with a membership degree of 0 is referred to as unfit for the class, while an object have a higher value has a better fit, with the value of 1 being a perfect fit into the class in question.
            \subsection{Partition}
               Let X be a set, and $A_1, A_2,...,A_n$ be subsets of sets such that $A_1 \cap A_2 \cap \dots \cap A_n = \emptyset$ and $A_1 \cup A_2 \cup \dots \cup A_n = X$. The disjoint sets $A_1, A_2, \dots, A_n$ are referred to as a \textit{partition} of the set X
			
			   Throughout this text, all mathematical notations retain their usual meanings, unless otherwise stated.
	\chapter{Literature Review}
		\paragraph{In}
			this section we shall proceed to show some theoretical framework for this work.
		\section{Linear Programming (LP)}
			Although not used in this study, it bears mentioningAs mentioned above, Linear Programming techniques often lend themselves to making optimum resource allocation in many cases. The following definition accurately captures these techniques:
				\begin{verse}
					\textit{
						\begin{center}
						Linear Programming is a mathematical technique for determining the optimal allocation of resources and obtaining a particular objective (i.e, cost minimization or inversely profit maximization) when there are alternative uses of resources: Land, Capital, Materials, Machines, etc \cite{stats}
						\end{center}
					}
				\end{verse}
			Solving an assignment problem with Linear Programming techniques, often requires a clear statement of an Objective function, resources constraints, usually expressed as a set of multivariate linear equations.
			\paragraph{}
				The general formulation of a Linear Programming problem is given below:
				\begin{verse}
					\begin{center}
						\textit{
							Let Z be a linear function defined by \\
							$Z = c_1x_1 + c_2x_2+...+c_nx_n$ \\
							where $c_j$'s are constants \\
							Let $(a_{ij})$ be mn constants and let $(b_i)$ be a set of m constants such that \\
							$a_{11}x_1+a_{12}x_2+...+a_{1n}x_n (\leq,=,\geq) b_1$ \\
							$a_{21}x_1+a_{22}x_2+...+a_{2n}x_n (\leq,=,\geq) b_2$ \\
							$\cdot$ \\
							$\cdot$ \\
							$\cdot$ \\
							$a_{m1}x_1+a_{m2}x_2+...+a_{mn}x_n (\leq,=,\geq) b_m$ \\
							$x_1\geq0,x_2\geq0,...,x_n\geq0$ \\
						}                   
					\end{center}
				\end{verse}
				The problem of determining the values of $x_1, x_2,...,x_n$, which makes Z a minimum ( or maximum ) and which satisfies (ii) and (iii) is called the General Linear Programming Problem    
			\paragraph{}
				Linear Programming techniques, as well as other Mathematical Optimization techniques have been reputed a great success, nonetheless, there are several yet unaddressed perspectives of optimum assignment problems
		\section{Fuzzy Set}
			Since its inception in 1965, the theory of fuzzy sets has advanced in a variety of ways and in many disciplines. Applications of this theory can be found, for example, in artificial intelligence, computer science, medicine, control engineering, decision theory, expert systems, logic, management science, operations research, pattern recognition, and robotics. 
			Mathematical developments have advanced to a very high standard and are still forthcoming to day. In this review, the basic mathematical framework of fuzzy set theory will be described.
			\\We begin with a definition of the Fuzzy Set Theory: \cite{fuz}
			\begin{verse}
				\textit{
					\begin{center}
						If X is a nonempty collection of objects denoted
						generically by $x_1,x_2,...,x_n$ for $n = 1$,~$2$, $\dots$, then a fuzzy set A in X is a pair \\$(A,\mu_A)$
						\\where $A \subset X $ and $\mu_A \colon \textbf{R} \to [0,1]$, called the \emph{membership function of A}
					\end{center}
				}
			\end{verse}
			\subsubsection{Examples of Fuzzy Sets}
				To better understand the concept of Fuzzy Sets, let us consider the following Examples
				\paragraph{Example 1}
					Let us define a fuzzy set 
					\[
					A = \{\, x \mid x \in R, \text{ real numbers near 0 } \,\}
					\]
					The boundary for set “real number near 0” is pretty ambiguous. The possibility of real number x to be a member of prescribed set can be defined by the following membership function.\\
					We define the membership function
					\begin{verse}
						$ \mu_A \colon \textbf{R} \to [0,1] $
					\end{verse}
					Defining the membership of elements of $\mathbb{R}$ in $\mathbf{A}$\\
					Defined as:
					\begin{verse}
						\begin{equation}
							\mu_A(x) = \frac{1}{1+x^2}
						\end{equation}
					\end{verse}
				\paragraph{Example 2}
					Let us define a similar fuzzy set 
					\[
					B = \{\, x \mid x \in R, \text{ real numbers very near 0 } \,\}
					\]
					\\We define the membership function
					\begin{verse}
						$\mu_B\colon \textbf{R} \to [0,1] $
					\end{verse}
					Defining the membership of elements of $\mathbb{R}$ in $\mathbf{B}$ such that:
					\begin{verse}
						\begin{equation}
							\mu_B(x) = \left( \frac{1}{1+x^2} \right)^{2}
						\end{equation}
					\end{verse}
				\paragraph{}
					The Theory of Fuzzy Sets forms the basic framework upon which the solution proposed by this work is formulated
	\chapter{The Algorithm}
		\section{Defining of Resources and Objects}
			\paragraph{In}
				this section we shall proceed to develop the algrithm needed to solve the reduction problem proposed in the previous section.
			\paragraph{Let the set $\mathbb(U)$} be a collection of objects to be assigned to partitioning sets
				\begin{equation}
					\mathbb{U} = \{u_1, u_2,\dots,u_n\}				
				\end{equation}
				where each $u_i i=1,2,\dots$,n are the objects possessing characteristics to be assigned to various partition classes, e.g say, the group of graduates to be posted to NYSC serving states
			\paragraph{And let}
			\begin{equation}
				p_1,p_2,\dots,p_k
			\end{equation}
				be the partitions to which objects $u_i$'s are to be assigned. We observe that there are $k$ such classes, eg, each $p_i$ would be a state where graduates can be posted to for their Service Year
		\section{Adjusting for Capacities}
			\paragraph{We now let}
				\begin{equation}
					J(p_i) = j_i
				\end{equation}
				be the maxiumum capacity the partition $p_i$ can contain. That is, the maximum number of $u_i$'s each partition $p_i$ can contain
			\paragraph{Let V,}
				\begin{equation}
					V = \sum_{ i = 1 }^{k}{J(p_i)} = \sum_{i=1}^{k}{j_i}
				\end{equation}
				be the total capacity of the system, that is, the total number of $u_i$'s all the partitions can contain combined.\\
				NB: It should be noted that for an optimum solution to be feasible, $\mid U \mid \leq V$ needs be satisfied
			\paragraph{We}
				now define the weight, $w_i$ of each partition $p_i$
				\begin{equation}
					w_i = \frac{j_i}{V} = \frac{J(p_i)}{\sum_{i=1}^{k}{J(p_i)}}  \textit{ for i=1,2,3,}\dots\text{,}k
				\end{equation}
				The weight, $w_i$ represents the proportion of each $p_i$'s capacity in the whole
			\paragraph{We}
				now define the expected cardinality, $\mid p_i \mid$ of each partition $p_i$
				\begin{equation}
					\mid p_i \mid = w_i \cdot \mid U \mid = w_i \cdot n, \textit{for i=1,2,3,}\dots\text{,}k
				\end{equation}
				The cardinality, $\mid p_i \mid$ represents the number of objects proportionately assignable to each partition $p_i$
		\section{Working with the Constraints}
			\paragraph{}
				We now proceed to define constraints functions, akin to membership function of a Fuzzy Set. But in this case, they membership functions are of two types: inclusion and exclusion types.\\
				Inclusion Constraints require that the value of a characteristic measured on candidate objects, $u_i$ match that prescribed by the constraints, while on the otherhand, Exclusion constraints define characteristics that once a candidate object $u_i$ posseses, it is disqualified ( by assigning a value of 0 ) from being a probable member of the partition in question\\
				Constraints, like membership functions, are generally of the form
				\[
					\mu \colon U \mapsto [ 0,1 ]
				\]
				
				Inclusion Constraints are defined thus:
				\begin{equation}
					\mu_j(u_i) = \frac{1}{1+\left( j - s(u_i) \right)^2}
				\end{equation}
				
				And, Exclusion Constraints are defined thus:
				\begin{equation}
					\mu_j(u_i) = \frac{\left( j - s(u_i) \right)^2}{1+\left( j - s(u_i) \right)^2}
				\end{equation}
				$\textit{for i=1,2,}\dots\text{,}n$ and $\textit{ j=1,2,}\dots\text{,}k$\\
				and
				\begin{equation}
					s\colon U\mapsto \{ 1,2,\dots,k \}
				\end{equation}
				is a helper function defined by the individual Constraint function, $\mu_j$ to get the index of the partition class $p_j$ with which each object $u_i$ shares the characteristic being measured\\
				For instance, if $s$ is the state of origin helper function, it returns the index of the partition class $p_i$ to which the object $u_i$ maps as a state of origin\\
				NB: $p_1, p_2,\dots,k$ are indexed based on proximity or similarity. That is, the difference between the indices of any two partitions is directly proportional to the distance between them.
			\paragraph{}
				Now, in situations where there are more than one Constraint functions defined, we obtain a single Constraint function by multiplying through viz:
				\begin{equation}
					\mu_i(u_j) = \prod_{t=1}^{s}{\mu_{i_t}(u_j)}
				\end{equation}
				$\textit{for i=1,2,}\dots\text{,}k$; $\textit{ j=1,2,}\dots\text{,}n$ and $s$ is the total number of Constraint functions defined
			\paragraph{}
				We now define the membership set for each partition. These sets list the probability of each object $u_i$ being a member of the partition $p_j$
				\begin{equation}
					\mu_i = \{ \mu_i(u_1), \mu_i(u_2),\dots,\mu_i(u_n) \}
				\end{equation}
				So, for example
				\begin{equation}
					\mu_1 = \{ \mu_1(u_1), \mu_1(u_2),\dots,\mu_1(u_n) \}
				\end{equation}
				and,
				\begin{equation}
					\mu_2 = \{ \mu_2(u_1), \mu_2(u_2),\dots,\mu_2(u_n) \}
				\end{equation}
				and,
				\begin{equation}
					\mu_3 = \{ \mu_3(u_1), \mu_3(u_2),\dots,\mu_3(u_n) \}
				\end{equation}
				\[
					\vdots
				\]
				\begin{equation}
					\mu_k = \{ \mu_k(u_1), \mu_k(u_2),\dots,\mu_k(u_n) \}
				\end{equation}
		\section{Assigning the Objects}
			\paragraph{}
				We now proceed to populate each partition with the objects with the highest membership probability, in turn \\
				Let
				\begin{equation}
					\mu_{p_i} = \mu_i \backslash \bigcup_{j=1}^{u-1}{ \{ \mu_j(u_x) \forall u_x \in p_j \} }
				\end{equation}
				That is, isolating the already assigned objects $u_x$ from the membership set to avoid multiple membership
			\paragraph{}
				Now, sorting out the first $\mid p_k \mid$ objects with the highest membership points:\\
				Let
					\begin{equation}
						Q_1 = u_x \mid \mu_i(u_x) = \text{max}\left\{ \mu_{p_i} \right\}
					\end{equation}
				And
					\begin{equation}
						Q_2 = u_x \mid \mu_i(u_x) = \text{max}\left\{ \mu_{p_i} \backslash \mu_i(u_y) \mid u_y \in Q_1 \right\}
					\end{equation}
				And
					\begin{equation}
						Q_3 = u_x \mid \mu_i(u_x) = \text{max}\left\{ \mu_{p_i} \backslash \mu_i(u_y) \mid u_y \in Q_1 \bigcup Q_2 \right\}
					\end{equation}
				And
					\begin{equation}
						Q_4 = u_x \mid \mu_i(u_x) = \text{max}\left\{ \mu_{p_i} \backslash \mu_i(u_y) \mid u_y \in Q_1 \bigcup Q_2 \bigcup Q_3 \right\}
					\end{equation}
					\[
						\vdots
					\]
					\begin{equation}
						Q_{\mid p_i \mid} = u_x \mid \mu_i(u_x) = \text{max}\left\{ \mu_{p_i} \backslash \bigcup_{j=1}^{\mid p_i \mid - 1} \left\{  \mu_i(u_y) \mid u_y \in Q_j \right\} \right\}
					\end{equation}
				Whence we finally have that
					\begin{equation}
						p_i = \bigcup_{j=1}^{\mid p_i \mid}\left\{Q_j\right\} \qquad \textit{for i=1,2,}\dots\text{,}k
					\end{equation}
	\chapter{Solving A Sample Problem}
		\paragraph{In}
			this section, we attempt to apply the algorithm developed in the previous chapter to solve a sample problem. The problem is an attempt to assign a group of 20 University graduates to one of 5 states for service.
		
		\section{Statement of Constraints}
			\begin{enumerate}
				\item No students should be posted to their State of Origin
				\item No students should be posted to their States of study
				\item The maximum capacity of each state must not be exceeded
				\item All students should be assigned
			\end{enumerate}
			The 3rd and 4th constraints are inately adhered to by the algorithm, so we'll focus only on constructing Constraint Functions for the first two constraints.
		\section{The States}
			\paragraph{}
				The States in this problem stand for the partitions to be filled. In order to simplify the problem, we focus only on 5 states and assume each of the 20 graduates schooled in one of these and originated from one of these states. The data represented here is in the form \{State,Capacity\}
				\begin{enumerate}
					\item $p_1\colon$ \{Osun State, 3\}
					\item $p_2\colon$ \{Kwara State, 2\}
					\item $p_3\colon$ \{Enugu State, 1\}
					\item $p_4\colon$ \{Akwa-Ibom State, 5\}
					\item $p_5\colon$ \{Kano State, 10\}
				\end{enumerate}
		\section{The Graduates}
			\paragraph{}
				Each graduate has a number of characteristics which include names, State of origin, and Alma Mater. Here is our list of 20 graduates which we shall consider in this sample problem:
				\begin{enumerate}
					\item $u_1\colon$ \{GR01, Kwara(2), OAU(1)\}
					\item $u_2\colon$ \{GR02, Kwara(2), BUK(5)\}
					\item $u_3\colon$ \{GR03, Kwara(2), NSUKKA(3)\}
					\item $u_4\colon$ \{GR04, Kwara(2), UNILORIN(2)\}
					\item $u_5\colon$ \{GR05, Akwa-Ibom(4), OAU(1)\}
					\item $u_6\colon$ \{GR06, Akwa-Ibom(4), BUK(5)\}
					\item $u_7\colon$ \{GR07, Akwa-Ibom(4), NSUKKA(3)\}
					\item $u_8\colon$ \{GR08, Akwa-Ibom(4), UNILORIN(2)\}
					\item $u_9\colon$ \{GR09, Osun(1), OAU(1)\}
					\item $u_{10}\colon$ \{GR10, Osun(1), BUK(5)\}
					\item $u_{11}\colon$ \{GR11, Osun(1), NSUKKA(3)\}
					\item $u_{12}\colon$ \{GR12, Osun(1), UNILORIN(2)\}
					\item $u_{13}\colon$ \{GR13, Kano(5), OUI(1)\}
					\item $u_{14}\colon$ \{GR14, Kano(5), NSUKKA(3)\}
					\item $u_{15}\colon$ \{GR15, Kano(5), UNILORIN(2)\}
					\item $u_{16}\colon$ \{GR16, Kano(5), AKSU(4)\}
					\item $u_{17}\colon$ \{GR17, Enugu(3), UNIOSUN(1)\}
					\item $u_{18}\colon$ \{GR18, Enugu(3), BUK(5)\}
					\item $u_{19}\colon$ \{GR19, Enugu(3), UNILORIN(2)\}
					\item $u_{20}\colon$ \{GR20, Enugu(3), AKSU(4)\}
				\end{enumerate}
		\section{States - Capacity, Weights and Cardinality}
			\paragraph{}
				As indicated in the States data, each state has a capacity, $j_i$. So in this case,
				\begin{itemize}
					\item[ ] \[J(p_1) = j_1 = 3\]
					\item[ ] \[J(p_2) = j_2 = 2\]
					\item[ ] \[J(p_3) = j_3 = 1\]
					\item[ ] \[J(p_4) = j_4 = 5\]
					\item[ ] \[J(p_5) = j_5 = 10\]
				\end{itemize}
				Thus, the maximum capacity of the system is given as:
				\begin{equation}
					V = \sum_{i=1}^{5}{J(p_i)} = 3 + 2 + 1 + 5 + 10 = 21
				\end{equation}
			Since $V$ is less than the total number of graduates awaiting deployment (20), a feasible assignment is possible.
			\paragraph{}
				The weight, $w_i$ of each state is defined thus:
				\begin{equation}
					w_i = \frac{j_i}{V}, \qquad \textit{for i=1,2,...,5}
				\end{equation}
				\begin{equation}
					w_1 = \frac{j_1}{V} = \frac{3}{21} = 0.143
				\end{equation}
				\begin{equation}
					w_2 = \frac{j_2}{V} = \frac{2}{21} = 0.095
				\end{equation}
				\begin{equation}
					w_3 = \frac{j_3}{V} = \frac{1}{21} = 0.048
				\end{equation}
				\begin{equation}
					w_4 = \frac{j_4}{V} = \frac{5}{21} = 0.238
				\end{equation}
				\begin{equation}
					w_5 = \frac{j_5}{V} = \frac{10}{21} = 0.476
				\end{equation}
			\paragraph{}
				We now define the expected cardinality, $\mid p_i\mid$ of each state. That is, the expected number of Corp members posted to each state
				\begin{align}
					\mid p_i \mid = w_i \cdot \text{no. of graduates} = w_i \cdot n = w_i \cdot 20, \qquad \textit{for i=1,2,...,5}
				\end{align}
				\begin{align}
					\mid p_1 \mid &= w_1 \times 20 = 0.143 \times 20 = 2.86 \approx 3 \\
					\mid p_2 \mid &= w_2 \times 20 = 0.095 \times 20 = 1.9 \approx 2 \\
					\mid p_3 \mid &= w_3 \times 20 = 0.048 \times 20 = 0.96 \approx 1 \\
					\mid p_4 \mid &= w_4 \times 20 = 0.238 \times 20 = 4.76 \approx 5 \\
					\mid p_5 \mid &= w_5 \times 20 = 0.476 \times 20 = 9.52 \approx 9
				\end{align}
		\section{Constraints}
			The sample problem has two statements of exclusion constraints. We will establish the Constraint functions defining the membership probability of each graduate into each state thus:
			\paragraph{Constraint A, $\mu_A$: No student should be posted to their State of Origin}
				\begin{equation}
					\mu_{A_j}(u_i) = \frac{\left(j-s_A(u_i)\right)^2}{1 + \left(j-s_A(u_i)\right)^2}
				\end{equation}
				where $\mu_{A_j}(u_i)$ measures the degree of object $u_i$'s membership in State $j$ under the consideration of the \textit{state of origin} constraint and,
				$s_A(u_i)$ represents the index of graduate $u_i$'s State of Origin
				\begin{align}
					\mu_{A_1}(u_1) &= \frac{\left(1-s_A(u_1)\right)^2}{1 + \left(1-s_A(u_1)\right)^2}\\
					&= \frac{\left(1-2\right)^2}{1 + \left(1-2\right)^2} \notag \\
					&= 0.5 \notag
				\end{align}
				\begin{align}
					\mu_{A_1}(u_2) &= \frac{\left(1-s_A(u_2)\right)^2}{1 + \left(1-s_A(u_2)\right)^2}\\
					&= \frac{\left(1-2\right)^2}{1 + \left(1-2\right)^2} \notag \\
					&= 0.5 \notag
				\end{align}
				\begin{align}
					\mu_{A_1}(u_3) &= \frac{\left(1-s_A(u_3)\right)^2}{1 + \left(1-s_A(u_3)\right)^2}\\
					&= \frac{\left(1-2\right)^2}{1 + \left(1-2\right)^2} \notag \\
					&= 0.5 \notag
				\end{align}
				\begin{align}
					\mu_{A_1}(u_4) &= \frac{\left(1-s_A(u_4)\right)^2}{1 + \left(1-s_A(u_4)\right)^2}\\
					&= \frac{\left(1-2\right)^2}{1 + \left(1-2\right)^2} \notag \\
					&= 0.5 \notag
				\end{align}
				\begin{align}
					\mu_{A_1}(u_5) &= \frac{\left(1-s_A(u_5)\right)^2}{1 + \left(1-s_A(u_5)\right)^2}\\
					&= \frac{\left(1-4\right)^2}{1 + \left(1-4\right)^2} \notag \\
					&= 0.9 \notag
				\end{align}
				\begin{align}
					\mu_{A_1}(u_6) &= \frac{\left(1-s_A(u_6)\right)^2}{1 + \left(1-s_A(u_6)\right)^2}\\
					&= \frac{\left(1-4\right)^2}{1 + \left(1-4\right)^2} \notag \\
					&= 0.9 \notag
				\end{align}
				\begin{align}
					\mu_{A_1}(u_7) &= \frac{\left(1-s_A(u_7)\right)^2}{1 + \left(1-s_A(u_7)\right)^2}\\
					&= \frac{\left(1-4\right)^2}{1 + \left(1-4\right)^2} \notag \\
					&= 0.9 \notag
				\end{align}
				\begin{align}
					\mu_{A_1}(u_8) &= \frac{\left(1-s_A(u_8)\right)^2}{1 + \left(1-s_A(u_8)\right)^2}\\
					&= \frac{\left(1-4\right)^2}{1 + \left(1-4\right)^2} \notag \\
					&= 0.9 \notag
				\end{align}
				\begin{align}
					\mu_{A_1}(u_9) &= \frac{\left(1-s_A(u_9)\right)^2}{1 + \left(1-s_A(u_9)\right)^2}\\
					&= \frac{\left(1-1\right)^2}{1 + \left(1-1\right)^2} \notag \\
					&= 0 \notag
				\end{align}
				\begin{align}
					\mu_{A_1}(u_{10}) &= \frac{\left(1-s_A(u_{10})\right)^2}{1 + \left(1-s_A(u_{10})\right)^2}\\
					&= \frac{\left(1-1\right)^2}{1 + \left(1-1\right)^2} \notag \\
					&= 0 \notag
				\end{align}
				\begin{align}
					\mu_{A_1}(u_{11}) &= \frac{\left(1-s_A(u_{11})\right)^2}{1 + \left(1-s_A(u_{11})\right)^2}\\
					&= \frac{\left(1-1\right)^2}{1 + \left(1-1\right)^2} \notag \\
					&= 0 \notag
				\end{align}
				\begin{align}
					\mu_{A_1}(u_{12}) &= \frac{\left(1-s_A(u_{12})\right)^2}{1 + \left(1-s_A(u_{12})\right)^2}\\
					&= \frac{\left(1-1\right)^2}{1 + \left(1-1\right)^2} \notag \\
					&= 0 \notag
				\end{align}
				\begin{align}
					\mu_{A_1}(u_{13}) &= \frac{\left(1-s_A(u_{13})\right)^2}{1 + \left(1-s_A(u_{13})\right)^2}\\
					&= \frac{\left(1-5\right)^2}{1 + \left(1-5\right)^2} \notag \\
					&= 0.94 \notag
				\end{align}
				\begin{align}
					\mu_{A_1}(u_{14}) &= \frac{\left(1-s_A(u_{14})\right)^2}{1 + \left(1-s_A(u_{14})\right)^2}\\
					&= \frac{\left(1-5\right)^2}{1 + \left(1-5\right)^2} \notag \\
					&= 0.94 \notag
				\end{align}
				\begin{align}
					\mu_{A_1}(u_{15}) &= \frac{\left(1-s_A(u_{15})\right)^2}{1 + \left(1-s_A(u_{15})\right)^2}\\
					&= \frac{\left(1-5\right)^2}{1 + \left(1-5\right)^2} \notag \\
					&= 0.94 \notag
				\end{align}
				\begin{align}
					\mu_{A_1}(u_{16}) &= \frac{\left(1-s_A(u_{16})\right)^2}{1 + \left(1-s_A(u_{16})\right)^2}\\
					&= \frac{\left(1-5\right)^2}{1 + \left(1-5\right)^2} \notag \\
					&= 0.94 \notag
				\end{align}
				\begin{align}
					\mu_{A_1}(u_{17}) &= \frac{\left(1-s_A(u_{17})\right)^2}{1 + \left(1-s_A(u_{17})\right)^2}\\
					&= \frac{\left(1-3\right)^2}{1 + \left(1-3\right)^2} \notag \\
					&= 0.8 \notag
				\end{align}
				\begin{align}
					\mu_{A_1}(u_{18}) &= \frac{\left(1-s_A(u_{18})\right)^2}{1 + \left(1-s_A(u_{18})\right)^2}\\
					&= \frac{\left(1-3\right)^2}{1 + \left(1-3\right)^2} \notag \\
					&= 0.8 \notag
				\end{align}
				\begin{align}
					\mu_{A_1}(u_{19}) &= \frac{\left(1-s_A(u_{19})\right)^2}{1 + \left(1-s_A(u_{19})\right)^2}\\
					&= \frac{\left(1-3\right)^2}{1 + \left(1-3\right)^2} \notag \\
					&= 0.8 \notag
				\end{align}
				\[
					\vdots
				\]
				\begin{align}
					\mu_{A_1}(u_{20}) &= \frac{\left(1-s_A(u_{20})\right)^2}{1 + \left(1-s_A(u_{20})\right)^2}\\
					&= \frac{\left(1-3\right)^2}{1 + \left(1-3\right)^2} \notag \\
					&= 0.8 \notag
				\end{align}
				% \\
				% \\
				% \\
				% \[
				% 	\vdots
				% \]
				\begin{align}
					\mu_{A_2}(u_1) &= \frac{\left(2-s_A(u_1)\right)^2}{1 + \left(2-s_A(u_1)\right)^2}\\
					&= \frac{\left(2-2\right)^2}{1 + \left(2-2\right)^2} \notag \\
					&= 0 \notag
				\end{align}
				\begin{align}
					\mu_{A_2}(u_2) &= \frac{\left(2-s_A(u_2)\right)^2}{1 + \left(2-s_A(u_2)\right)^2}\\
					&= \frac{\left(2-2\right)^2}{1 + \left(2-2\right)^2} \notag \\
					&= 0 \notag
				\end{align}
				\begin{align}
					\mu_{A_2}(u_3) &= \frac{\left(2-s_A(u_3)\right)^2}{1 + \left(2-s_A(u_3)\right)^2}\\
					&= \frac{\left(2-2\right)^2}{1 + \left(2-2\right)^2} \notag \\
					&= 0 \notag
				\end{align}
				\begin{align}
					\mu_{A_2}(u_4) &= \frac{\left(2-s_A(u_4)\right)^2}{1 + \left(2-s_A(u_4)\right)^2}\\
					&= \frac{\left(2-2\right)^2}{1 + \left(2-2\right)^2} \notag \\
					&= 0 \notag
				\end{align}
				\begin{align}
					\mu_{A_2}(u_5) &= \frac{\left(2-s_A(u_5)\right)^2}{1 + \left(2-s_A(u_5)\right)^2}\\
					&= \frac{\left(2-4\right)^2}{1 + \left(2-4\right)^2} \notag \\
					&= 0.8 \notag
				\end{align}
				\begin{align}
					\mu_{A_2}(u_6) &= \frac{\left(2-s_A(u_6)\right)^2}{1 + \left(2-s_A(u_6)\right)^2}\\
					&= \frac{\left(2-4\right)^2}{1 + \left(2-4\right)^2} \notag \\
					&= 0.8 \notag
				\end{align}
				\begin{align}
					\mu_{A_2}(u_7) &= \frac{\left(2-s_A(u_7)\right)^2}{1 + \left(2-s_A(u_7)\right)^2}\\
					&= \frac{\left(2-4\right)^2}{1 + \left(2-4\right)^2} \notag \\
					&= 0.8 \notag
				\end{align}
				\begin{align}
					\mu_{A_2}(u_8) &= \frac{\left(2-s_A(u_8)\right)^2}{1 + \left(2-s_A(u_8)\right)^2}\\
					&= \frac{\left(2-4\right)^2}{1 + \left(2-4\right)^2} \notag \\
					&= 0.8 \notag
				\end{align}
				\begin{align}
					\mu_{A_2}(u_9) &= \frac{\left(2-s_A(u_9)\right)^2}{1 + \left(2-s_A(u_9)\right)^2}\\
					&= \frac{\left(2-1\right)^2}{1 + \left(2-1\right)^2} \notag \\
					&= 0.5 \notag
				\end{align}
				\begin{align}
					\mu_{A_2}(u_{10}) &= \frac{\left(2-s_A(u_{10})\right)^2}{1 + \left(2-s_A(u_{10})\right)^2}\\
					&= \frac{\left(2-1\right)^2}{1 + \left(2-1\right)^2} \notag \\
					&= 0.5 \notag
				\end{align}
				\begin{align}
					\mu_{A_2}(u_{11}) &= \frac{\left(2-s_A(u_{11})\right)^2}{1 + \left(2-s_A(u_{11})\right)^2}\\
					&= \frac{\left(2-1\right)^2}{1 + \left(2-1\right)^2} \notag \\
					&= 0.5 \notag
				\end{align}
				\begin{align}
					\mu_{A_2}(u_{12}) &= \frac{\left(2-s_A(u_{12})\right)^2}{1 + \left(2-s_A(u_{12})\right)^2}\\
					&= \frac{\left(2-1\right)^2}{1 + \left(2-1\right)^2} \notag \\
					&= 0.5 \notag
				\end{align}
				\begin{align}
					\mu_{A_2}(u_{13}) &= \frac{\left(2-s_A(u_{13})\right)^2}{1 + \left(2-s_A(u_{13})\right)^2}\\
					&= \frac{\left(2-5\right)^2}{1 + \left(2-5\right)^2} \notag \\
					&= 0.90 \notag
				\end{align}
				\begin{align}
					\mu_{A_2}(u_{14}) &= \frac{\left(2-s_A(u_{14})\right)^2}{1 + \left(2-s_A(u_{14})\right)^2}\\
					&= \frac{\left(2-5\right)^2}{1 + \left(2-5\right)^2} \notag \\
					&= 0.90 \notag
				\end{align}
				\begin{align}
					\mu_{A_2}(u_{15}) &= \frac{\left(2-s_A(u_{15})\right)^2}{1 + \left(2-s_A(u_{15})\right)^2}\\
					&= \frac{\left(2-5\right)^2}{1 + \left(2-5\right)^2} \notag \\
					&= 0.90 \notag
				\end{align}
				\begin{align}
					\mu_{A_2}(u_{16}) &= \frac{\left(2-s_A(u_{16})\right)^2}{1 + \left(2-s_A(u_{16})\right)^2}\\
					&= \frac{\left(2-5\right)^2}{1 + \left(2-5\right)^2} \notag \\
					&= 0.90 \notag
				\end{align}
				\begin{align}
					\mu_{A_2}(u_{17}) &= \frac{\left(2-s_A(u_{17})\right)^2}{1 + \left(2-s_A(u_{17})\right)^2}\\
					&= \frac{\left(2-3\right)^2}{1 + \left(2-3\right)^2} \notag \\
					&= 0.50 \notag
				\end{align}
				\begin{align}
					\mu_{A_2}(u_{18}) &= \frac{\left(2-s_A(u_{18})\right)^2}{1 + \left(2-s_A(u_{18})\right)^2}\\
					&= \frac{\left(2-3\right)^2}{1 + \left(2-3\right)^2} \notag \\
					&= 0.50 \notag
				\end{align}
				\begin{align}
					\mu_{A_2}(u_{19}) &= \frac{\left(2-s_A(u_{19})\right)^2}{1 + \left(2-s_A(u_{19})\right)^2}\\
					&= \frac{\left(2-3\right)^2}{1 + \left(2-3\right)^2} \notag \\
					&= 0.50 \notag
				\end{align}
				% \[
				% 	\vdots
				% \]
				\begin{align}
					\mu_{A_2}(u_{20}) &= \frac{\left(2-s_A(u_{20})\right)^2}{1 + \left(2-s_A(u_{20})\right)^2}\\
					&= \frac{\left(2-3\right)^2}{1 + \left(2-3\right)^2} \notag \\
					&= 0.50 \notag
				\end{align}
				% \[
				% 	\vdots
				% \]
				\begin{align}
					\mu_{A_3}(u_1) &= \frac{\left(3-s_A(u_1)\right)^2}{1 + \left(3-s_A(u_1)\right)^2}\\
					&= \frac{\left(3-2\right)^2}{1 + \left(3-2\right)^2} \notag \\
					&= 0.5 \notag
				\end{align}
				\begin{align}
					\mu_{A_3}(u_2) &= \frac{\left(3-s_A(u_2)\right)^2}{1 + \left(3-s_A(u_2)\right)^2}\\
					&= \frac{\left(3-2\right)^2}{1 + \left(3-2\right)^2} \notag \\
					&= 0.5 \notag
				\end{align}
				\begin{align}
					\mu_{A_3}(u_3) &= \frac{\left(3-s_A(u_3)\right)^2}{1 + \left(3-s_A(u_3)\right)^2}\\
					&= \frac{\left(3-2\right)^2}{1 + \left(3-2\right)^2} \notag \\
					&= 0.5 \notag
				\end{align}
				\begin{align}
					\mu_{A_3}(u_4) &= \frac{\left(3-s_A(u_4)\right)^2}{1 + \left(3-s_A(u_4)\right)^2}\\
					&= \frac{\left(3-2\right)^2}{1 + \left(3-2\right)^2} \notag \\
					&= 0.5 \notag
				\end{align}
				\begin{align}
					\mu_{A_3}(u_5) &= \frac{\left(3-s_A(u_5)\right)^2}{1 + \left(3-s_A(u_5)\right)^2}\\
					&= \frac{\left(3-4\right)^2}{1 + \left(3-4\right)^2} \notag \\
					&= 0.5 \notag
				\end{align}
				\begin{align}
					\mu_{A_3}(u_6) &= \frac{\left(3-s_A(u_6)\right)^2}{1 + \left(3-s_A(u_6)\right)^2}\\
					&= \frac{\left(3-4\right)^2}{1 + \left(3-4\right)^2} \notag \\
					&= 0.5 \notag
				\end{align}
				\begin{align}
					\mu_{A_3}(u_7) &= \frac{\left(3-s_A(u_7)\right)^2}{1 + \left(3-s_A(u_7)\right)^2}\\
					&= \frac{\left(3-4\right)^2}{1 + \left(3-4\right)^2} \notag \\
					&= 0.5 \notag
				\end{align}
				\begin{align}
					\mu_{A_3}(u_8) &= \frac{\left(3-s_A(u_8)\right)^2}{1 + \left(3-s_A(u_8)\right)^2}\\
					&= \frac{\left(3-4\right)^2}{1 + \left(3-4\right)^2} \notag \\
					&= 0.5 \notag
				\end{align}
				\begin{align}
					\mu_{A_3}(u_9) &= \frac{\left(3-s_A(u_9)\right)^2}{1 + \left(3-s_A(u_9)\right)^2}\\
					&= \frac{\left(3-1\right)^2}{1 + \left(3-1\right)^2} \notag \\
					&= 0.8 \notag
				\end{align}
				\begin{align}
					\mu_{A_3}(u_{10}) &= \frac{\left(3-s_A(u_{10})\right)^2}{1 + \left(3-s_A(u_{10})\right)^2}\\
					&= \frac{\left(3-1\right)^2}{1 + \left(3-1\right)^2} \notag \\
					&= 0.8 \notag
				\end{align}
				\begin{align}
					\mu_{A_3}(u_{11}) &= \frac{\left(3-s_A(u_{11})\right)^2}{1 + \left(3-s_A(u_{11})\right)^2}\\
					&= \frac{\left(3-1\right)^2}{1 + \left(3-1\right)^2} \notag \\
					&= 0.8 \notag
				\end{align}
				\begin{align}
					\mu_{A_3}(u_{12}) &= \frac{\left(3-s_A(u_{12})\right)^2}{1 + \left(3-s_A(u_{12})\right)^2}\\
					&= \frac{\left(3-1\right)^2}{1 + \left(3-1\right)^2} \notag \\
					&= 0.8 \notag
				\end{align}
				\begin{align}
					\mu_{A_3}(u_{13}) &= \frac{\left(3-s_A(u_{13})\right)^2}{1 + \left(3-s_A(u_{13})\right)^2}\\
					&= \frac{\left(3-5\right)^2}{1 + \left(3-5\right)^2} \notag \\
					&= 0.80 \notag
				\end{align}
				\begin{align}
					\mu_{A_3}(u_{14}) &= \frac{\left(3-s_A(u_{14})\right)^2}{1 + \left(3-s_A(u_{14})\right)^2}\\
					&= \frac{\left(3-5\right)^2}{1 + \left(3-5\right)^2} \notag \\
					&= 0.80 \notag
				\end{align}
				\begin{align}
					\mu_{A_3}(u_{15}) &= \frac{\left(3-s_A(u_{15})\right)^2}{1 + \left(3-s_A(u_{15})\right)^2}\\
					&= \frac{\left(3-5\right)^2}{1 + \left(3-5\right)^2} \notag \\
					&= 0.80 \notag
				\end{align}
				\begin{align}
					\mu_{A_3}(u_{16}) &= \frac{\left(3-s_A(u_{16})\right)^2}{1 + \left(3-s_A(u_{16})\right)^2}\\
					&= \frac{\left(3-5\right)^2}{1 + \left(3-5\right)^2} \notag \\
					&= 0.80 \notag
				\end{align}
				\begin{align}
					\mu_{A_3}(u_{17}) &= \frac{\left(3-s_A(u_{17})\right)^2}{1 + \left(3-s_A(u_{17})\right)^2}\\
					&= \frac{\left(3-3\right)^2}{1 + \left(3-3\right)^2} \notag \\
					&= 0 \notag
				\end{align}
				\begin{align}
					\mu_{A_3}(u_{18}) &= \frac{\left(3-s_A(u_{18})\right)^2}{1 + \left(3-s_A(u_{18})\right)^2}\\
					&= \frac{\left(3-3\right)^2}{1 + \left(3-3\right)^2} \notag \\
					&= 0 \notag
				\end{align}
				\begin{align}
					\mu_{A_3}(u_{19}) &= \frac{\left(3-s_A(u_{19})\right)^2}{1 + \left(3-s_A(u_{19})\right)^2}\\
					&= \frac{\left(3-3\right)^2}{1 + \left(3-3\right)^2} \notag \\
					&= 0 \notag
				\end{align}
				% \[
				% 	\vdots
				% \]
				\begin{align}
					\mu_{A_3}(u_{20}) &= \frac{\left(3-s_A(u_{20})\right)^2}{1 + \left(3-s_A(u_{20})\right)^2}\\
					&= \frac{\left(3-3\right)^2}{1 + \left(3-3\right)^2} \notag \\
					&= 0 \notag
				\end{align}
				% \[
				% 	\vdots
				% \]
				% \\
				% \\
				% \\
				\begin{align}
					\mu_{A_4}(u_1) &= \frac{\left(4-s_A(u_1)\right)^2}{1 + \left(4-s_A(u_1)\right)^2}\\
					&= \frac{\left(4-2\right)^2}{1 + \left(4-2\right)^2} \notag \\
					&= 0.8 \notag
				\end{align}
				\begin{align}
					\mu_{A_4}(u_2) &= \frac{\left(4-s_A(u_2)\right)^2}{1 + \left(4-s_A(u_2)\right)^2}\\
					&= \frac{\left(4-2\right)^2}{1 + \left(4-2\right)^2} \notag \\
					&= 0.8 \notag
				\end{align}
				\begin{align}
					\mu_{A_4}(u_3) &= \frac{\left(4-s_A(u_3)\right)^2}{1 + \left(4-s_A(u_3)\right)^2}\\
					&= \frac{\left(4-2\right)^2}{1 + \left(4-2\right)^2} \notag \\
					&= 0.8 \notag
				\end{align}
				\begin{align}
					\mu_{A_4}(u_4) &= \frac{\left(4-s_A(u_4)\right)^2}{1 + \left(4-s_A(u_4)\right)^2}\\
					&= \frac{\left(4-2\right)^2}{1 + \left(4-2\right)^2} \notag \\
					&= 0.8 \notag
				\end{align}
				\begin{align}
					\mu_{A_4}(u_5) &= \frac{\left(4-s_A(u_5)\right)^2}{1 + \left(4-s_A(u_5)\right)^2}\\
					&= \frac{\left(4-4\right)^2}{1 + \left(4-4\right)^2} \notag \\
					&= 0 \notag
				\end{align}
				\begin{align}
					\mu_{A_4}(u_6) &= \frac{\left(4-s_A(u_6)\right)^2}{1 + \left(4-s_A(u_6)\right)^2}\\
					&= \frac{\left(4-4\right)^2}{1 + \left(4-4\right)^2} \notag \\
					&= 0 \notag
				\end{align}
				\begin{align}
					\mu_{A_4}(u_7) &= \frac{\left(4-s_A(u_7)\right)^2}{1 + \left(4-s_A(u_7)\right)^2}\\
					&= \frac{\left(4-4\right)^2}{1 + \left(4-4\right)^2} \notag \\
					&= 0 \notag
				\end{align}
				\begin{align}
					\mu_{A_4}(u_8) &= \frac{\left(4-s_A(u_8)\right)^2}{1 + \left(4-s_A(u_8)\right)^2}\\
					&= \frac{\left(4-4\right)^2}{1 + \left(4-4\right)^2} \notag \\
					&= 0 \notag
				\end{align}
				\begin{align}
					\mu_{A_4}(u_9) &= \frac{\left(4-s_A(u_9)\right)^2}{1 + \left(4-s_A(u_9)\right)^2}\\
					&= \frac{\left(4-1\right)^2}{1 + \left(4-1\right)^2} \notag \\
					&= 0.9 \notag
				\end{align}
				\begin{align}
					\mu_{A_4}(u_{10}) &= \frac{\left(4-s_A(u_{10})\right)^2}{1 + \left(4-s_A(u_{10})\right)^2}\\
					&= \frac{\left(4-1\right)^2}{1 + \left(4-1\right)^2} \notag \\
					&= 0.9 \notag
				\end{align}
				\begin{align}
					\mu_{A_4}(u_{11}) &= \frac{\left(4-s_A(u_{11})\right)^2}{1 + \left(4-s_A(u_{11})\right)^2}\\
					&= \frac{\left(4-1\right)^2}{1 + \left(4-1\right)^2} \notag \\
					&= 0.9 \notag
				\end{align}
				\begin{align}
					\mu_{A_4}(u_{12}) &= \frac{\left(4-s_A(u_{12})\right)^2}{1 + \left(4-s_A(u_{12})\right)^2}\\
					&= \frac{\left(4-1\right)^2}{1 + \left(4-1\right)^2} \notag \\
					&= 0.9 \notag
				\end{align}
				\begin{align}
					\mu_{A_4}(u_{13}) &= \frac{\left(4-s_A(u_{13})\right)^2}{1 + \left(4-s_A(u_{13})\right)^2}\\
					&= \frac{\left(4-5\right)^2}{1 + \left(4-5\right)^2} \notag \\
					&= 0.50 \notag
				\end{align}
				\begin{align}
					\mu_{A_4}(u_{14}) &= \frac{\left(4-s_A(u_{14})\right)^2}{1 + \left(4-s_A(u_{14})\right)^2}\\
					&= \frac{\left(4-5\right)^2}{1 + \left(4-5\right)^2} \notag \\
					&= 0.50 \notag
				\end{align}
				\begin{align}
					\mu_{A_4}(u_{15}) &= \frac{\left(4-s_A(u_{15})\right)^2}{1 + \left(4-s_A(u_{15})\right)^2}\\
					&= \frac{\left(4-5\right)^2}{1 + \left(4-5\right)^2} \notag \\
					&= 0.50 \notag
				\end{align}
				\begin{align}
					\mu_{A_4}(u_{16}) &= \frac{\left(4-s_A(u_{16})\right)^2}{1 + \left(4-s_A(u_{16})\right)^2}\\
					&= \frac{\left(4-5\right)^2}{1 + \left(4-5\right)^2} \notag \\
					&= 0.50 \notag
				\end{align}
				\begin{align}
					\mu_{A_4}(u_{17}) &= \frac{\left(4-s_A(u_{17})\right)^2}{1 + \left(4-s_A(u_{17})\right)^2}\\
					&= \frac{\left(4-3\right)^2}{1 + \left(4-3\right)^2} \notag \\
					&= 0.50 \notag
				\end{align}
				\begin{align}
					\mu_{A_4}(u_{18}) &= \frac{\left(4-s_A(u_{18})\right)^2}{1 + \left(4-s_A(u_{18})\right)^2}\\
					&= \frac{\left(4-3\right)^2}{1 + \left(4-3\right)^2} \notag \\
					&= 0.50 \notag
				\end{align}
				\begin{align}
					\mu_{A_4}(u_{19}) &= \frac{\left(4-s_A(u_{19})\right)^2}{1 + \left(4-s_A(u_{19})\right)^2}\\
					&= \frac{\left(4-3\right)^2}{1 + \left(4-3\right)^2} \notag \\
					&= 0.50 \notag
				\end{align}
				\[
					\vdots
				\]
				\begin{align}
					\mu_{A_4}(u_{20}) &= \frac{\left(4-s_A(u_{20})\right)^2}{1 + \left(4-s_A(u_{20})\right)^2}\\
					&= \frac{\left(4-3\right)^2}{1 + \left(4-3\right)^2} \notag \\
					&= 0.50 \notag
				\end{align}
				% \[
				% 	\vdots
				% \]
				% \\
				% \\
				% \\
				\begin{align}
					\mu_{A_5}(u_1) &= \frac{\left(5-s_A(u_1)\right)^2}{1 + \left(5-s_A(u_1)\right)^2}\\
					&= \frac{\left(5-2\right)^2}{1 + \left(5-2\right)^2} \notag \\
					&= 0.9 \notag
				\end{align}
				\begin{align}
					\mu_{A_5}(u_2) &= \frac{\left(5-s_A(u_2)\right)^2}{1 + \left(5-s_A(u_2)\right)^2}\\
					&= \frac{\left(5-2\right)^2}{1 + \left(5-2\right)^2} \notag \\
					&= 0.9 \notag
				\end{align}
				\begin{align}
					\mu_{A_5}(u_3) &= \frac{\left(5-s_A(u_3)\right)^2}{1 + \left(5-s_A(u_3)\right)^2}\\
					&= \frac{\left(5-2\right)^2}{1 + \left(5-2\right)^2} \notag \\
					&= 0.9 \notag
				\end{align}
				\begin{align}
					\mu_{A_5}(u_4) &= \frac{\left(5-s_A(u_4)\right)^2}{1 + \left(5-s_A(u_4)\right)^2}\\
					&= \frac{\left(5-2\right)^2}{1 + \left(5-2\right)^2} \notag \\
					&= 0.9 \notag
				\end{align}
				\begin{align}
					\mu_{A_5}(u_5) &= \frac{\left(5-s_A(u_5)\right)^2}{1 + \left(5-s_A(u_5)\right)^2}\\
					&= \frac{\left(5-4\right)^2}{1 + \left(5-4\right)^2} \notag \\
					&= 0.5 \notag
				\end{align}
				\begin{align}
					\mu_{A_5}(u_6) &= \frac{\left(5-s_A(u_6)\right)^2}{1 + \left(5-s_A(u_6)\right)^2}\\
					&= \frac{\left(5-4\right)^2}{1 + \left(5-4\right)^2} \notag \\
					&= 0.5 \notag
				\end{align}
				\begin{align}
					\mu_{A_5}(u_7) &= \frac{\left(5-s_A(u_7)\right)^2}{1 + \left(5-s_A(u_7)\right)^2}\\
					&= \frac{\left(5-4\right)^2}{1 + \left(5-4\right)^2} \notag \\
					&= 0.5 \notag
				\end{align}
				\begin{align}
					\mu_{A_5}(u_8) &= \frac{\left(5-s_A(u_8)\right)^2}{1 + \left(5-s_A(u_8)\right)^2}\\
					&= \frac{\left(5-4\right)^2}{1 + \left(5-4\right)^2} \notag \\
					&= 0.5 \notag
				\end{align}
				\begin{align}
					\mu_{A_5}(u_9) &= \frac{\left(5-s_A(u_9)\right)^2}{1 + \left(5-s_A(u_9)\right)^2}\\
					&= \frac{\left(5-1\right)^2}{1 + \left(5-1\right)^2} \notag \\
					&= 0.94 \notag
				\end{align}
				\begin{align}
					\mu_{A_5}(u_{10}) &= \frac{\left(5-s_A(u_{10})\right)^2}{1 + \left(5-s_A(u_{10})\right)^2}\\
					&= \frac{\left(5-1\right)^2}{1 + \left(5-1\right)^2} \notag \\
					&= 0.94 \notag
				\end{align}
				\begin{align}
					\mu_{A_5}(u_{11}) &= \frac{\left(5-s_A(u_{11})\right)^2}{1 + \left(5-s_A(u_{11})\right)^2}\\
					&= \frac{\left(5-1\right)^2}{1 + \left(5-1\right)^2} \notag \\
					&= 0.94 \notag
				\end{align}
				\begin{align}
					\mu_{A_5}(u_{12}) &= \frac{\left(5-s_A(u_{12})\right)^2}{1 + \left(5-s_A(u_{12})\right)^2}\\
					&= \frac{\left(5-1\right)^2}{1 + \left(5-1\right)^2} \notag \\
					&= 0.94 \notag
				\end{align}
				\begin{align}
					\mu_{A_5}(u_{13}) &= \frac{\left(5-s_A(u_{13})\right)^2}{1 + \left(5-s_A(u_{13})\right)^2}\\
					&= \frac{\left(5-5\right)^2}{1 + \left(5-5\right)^2} \notag \\
					&= 0 \notag
				\end{align}
				\begin{align}
					\mu_{A_5}(u_{14}) &= \frac{\left(5-s_A(u_{14})\right)^2}{1 + \left(5-s_A(u_{14})\right)^2}\\
					&= \frac{\left(5-5\right)^2}{1 + \left(5-5\right)^2} \notag \\
					&= 0 \notag
				\end{align}
				\begin{align}
					\mu_{A_5}(u_{15}) &= \frac{\left(5-s_A(u_{15})\right)^2}{1 + \left(5-s_A(u_{15})\right)^2}\\
					&= \frac{\left(5-5\right)^2}{1 + \left(5-5\right)^2} \notag \\
					&= 0 \notag
				\end{align}
				\begin{align}
					\mu_{A_5}(u_{16}) &= \frac{\left(5-s_A(u_{16})\right)^2}{1 + \left(5-s_A(u_{16})\right)^2}\\
					&= \frac{\left(5-5\right)^2}{1 + \left(5-5\right)^2} \notag \\
					&= 0 \notag
				\end{align}
				\begin{align}
					\mu_{A_5}(u_{17}) &= \frac{\left(5-s_A(u_{17})\right)^2}{1 + \left(5-s_A(u_{17})\right)^2}\\
					&= \frac{\left(5-3\right)^2}{1 + \left(5-3\right)^2} \notag \\
					&= 0.80 \notag
				\end{align}
				\begin{align}
					\mu_{A_5}(u_{18}) &= \frac{\left(5-s_A(u_{18})\right)^2}{1 + \left(5-s_A(u_{18})\right)^2}\\
					&= \frac{\left(5-3\right)^2}{1 + \left(5-3\right)^2} \notag \\
					&= 0.80 \notag
				\end{align}
				\begin{align}
					\mu_{A_5}(u_{19}) &= \frac{\left(5-s_A(u_{19})\right)^2}{1 + \left(5-s_A(u_{19})\right)^2}\\
					&= \frac{\left(5-3\right)^2}{1 + \left(5-3\right)^2} \notag \\
					&= 0.80 \notag
				\end{align}
				% \[
				% 	\vdots
				% \]
				\begin{align}
					\mu_{A_5}(u_{20}) &= \frac{\left(5-s_A(u_{20})\right)^2}{1 + \left(5-s_A(u_{20})\right)^2}\\
					&= \frac{\left(5-3\right)^2}{1 + \left(5-3\right)^2} \notag \\
					&= 0.80 \notag
				\end{align}
			The foregoings, based on the first constraint, calculates the degree of membership of each object(graduate) $u_i$ in each state
			\\
			In a similar manner, we proceed to calculate the degree of membership of each object(graduate) in each state.
			\paragraph{Constraint B, $\mu_B$: No student should be posted to their State of Studies}
				\begin{equation}
					\mu_{B_j}(u_i) = \frac{\left(j-s_B(u_i)\right)^2}{1 + \left(j-s_B(u_i)\right)^2}
				\end{equation}
				where $\mu_{B_j}(u_i)$ measures the degree of object $u_i$'s membership in State $j$ under the consideration of the \textit{state of origin} constraint and,
				$s_B(u_i)$ represents the index of graduate $u_i$'s State of Study\\
				\begin{align}
					\mu_{B_1}(u_1) &= \frac{\left(1-s_B(u_1)\right)^2}{1 + \left(1-s_B(u_1)\right)^2}\\
					&= \frac{\left(1-1\right)^2}{1 + \left(1-1\right)^2} \notag \\
					&= 0 \notag
				\end{align}
				\begin{align}
					\mu_{B_1}(u_2) &= \frac{\left(1-s_B(u_2)\right)^2}{1 + \left(1-s_B(u_2)\right)^2}\\
					&= \frac{\left(1-2\right)^5}{1 + \left(1-5\right)^2} \notag \\
					&= 0.94 \notag
				\end{align}
				\begin{align}
					\mu_{B_1}(u_3) &= \frac{\left(1-s_B(u_3)\right)^2}{1 + \left(1-s_B(u_3)\right)^2}\\
					&= \frac{\left(1-3\right)^2}{1 + \left(1-3\right)^2} \notag \\
					&= 0.80 \notag
				\end{align}
				\begin{align}
					\mu_{B_1}(u_4) &= \frac{\left(1-s_B(u_4)\right)^2}{1 + \left(1-s_B(u_4)\right)^2}\\
					&= \frac{\left(1-2\right)^2}{1 + \left(1-2\right)^2} \notag \\
					&= 0.5 \notag
				\end{align}
				\begin{align}
					\mu_{B_1}(u_5) &= \frac{\left(1-s_B(u_5)\right)^2}{1 + \left(1-s_B(u_5)\right)^2}\\
					&= \frac{\left(1-1\right)^2}{1 + \left(1-1\right)^2} \notag \\
					&= 0 \notag
				\end{align}
				\begin{align}
					\mu_{B_1}(u_6) &= \frac{\left(1-s_B(u_6)\right)^2}{1 + \left(1-s_B(u_6)\right)^2}\\
					&= \frac{\left(1-5\right)^2}{1 + \left(1-5\right)^2} \notag \\
					&= 0.94 \notag
				\end{align}
				\begin{align}
					\mu_{B_1}(u_7) &= \frac{\left(1-s_B(u_7)\right)^2}{1 + \left(1-s_B(u_7)\right)^2}\\
					&= \frac{\left(1-3\right)^2}{1 + \left(1-3\right)^2} \notag \\
					&= 0.80 \notag
				\end{align}
				\begin{align}
					\mu_{B_1}(u_8) &= \frac{\left(1-s_B(u_8)\right)^2}{1 + \left(1-s_B(u_8)\right)^2}\\
					&= \frac{\left(1-2\right)^2}{1 + \left(1-2\right)^2} \notag \\
					&= 0.50 \notag
				\end{align}
				\begin{align}
					\mu_{B_1}(u_9) &= \frac{\left(1-s_B(u_9)\right)^2}{1 + \left(1-s_B(u_9)\right)^2}\\
					&= \frac{\left(1-1\right)^2}{1 + \left(1-1\right)^2} \notag \\
					&= 0 \notag
				\end{align}
				\begin{align}
					\mu_{B_1}(u_{10}) &= \frac{\left(1-s_B(u_{10})\right)^2}{1 + \left(1-s_B(u_{10})\right)^2}\\
					&= \frac{\left(1-5\right)^2}{1 + \left(1-5\right)^2} \notag \\
					&= 0.94 \notag
				\end{align}
				\begin{align}
					\mu_{B_1}(u_{11}) &= \frac{\left(1-s_B(u_{11})\right)^2}{1 + \left(1-s_B(u_{11})\right)^2}\\
					&= \frac{\left(1-3\right)^2}{1 + \left(1-3\right)^2} \notag \\
					&= 0.80 \notag
				\end{align}
				\begin{align}
					\mu_{B_1}(u_{12}) &= \frac{\left(1-s_B(u_{12})\right)^2}{1 + \left(1-s_B(u_{12})\right)^2}\\
					&= \frac{\left(1-2\right)^2}{1 + \left(1-2\right)^2} \notag \\
					&= 0.50 \notag
				\end{align}
				\begin{align}
					\mu_{B_1}(u_{13}) &= \frac{\left(1-s_B(u_{13})\right)^2}{1 + \left(1-s_B(u_{13})\right)^2}\\
					&= \frac{\left(1-1\right)^2}{1 + \left(1-1\right)^2} \notag \\
					&= 0 \notag
				\end{align}
				\begin{align}
					\mu_{B_1}(u_{14}) &= \frac{\left(1-s_B(u_{14})\right)^2}{1 + \left(1-s_B(u_{14})\right)^2}\\
					&= \frac{\left(1-3\right)^2}{1 + \left(1-3\right)^2} \notag \\
					&= 0.80 \notag
				\end{align}
				\begin{align}
					\mu_{B_1}(u_{15}) &= \frac{\left(1-s_B(u_{15})\right)^2}{1 + \left(1-s_B(u_{15})\right)^2}\\
					&= \frac{\left(1-2\right)^2}{1 + \left(1-2\right)^2} \notag \\
					&= 0.50 \notag
				\end{align}
				\begin{align}
					\mu_{B_1}(u_{16}) &= \frac{\left(1-s_B(u_{16})\right)^2}{1 + \left(1-s_B(u_{16})\right)^2}\\
					&= \frac{\left(1-4\right)^2}{1 + \left(1-4\right)^2} \notag \\
					&= 0.90 \notag
				\end{align}
				\begin{align}
					\mu_{B_1}(u_{17}) &= \frac{\left(1-s_B(u_{17})\right)^2}{1 + \left(1-s_B(u_{17})\right)^2}\\
					&= \frac{\left(1-1\right)^2}{1 + \left(1-1\right)^2} \notag \\
					&= 0 \notag
				\end{align}
				\begin{align}
					\mu_{B_1}(u_{18}) &= \frac{\left(1-s_B(u_{18})\right)^2}{1 + \left(1-s_B(u_{18})\right)^2}\\
					&= \frac{\left(1-5\right)^2}{1 + \left(1-5\right)^2} \notag \\
					&= 0.94 \notag
				\end{align}
				\begin{align}
					\mu_{B_1}(u_{19}) &= \frac{\left(1-s_B(u_{19})\right)^2}{1 + \left(1-s_B(u_{19})\right)^2}\\
					&= \frac{\left(1-2\right)^2}{1 + \left(1-2\right)^2} \notag \\
					&= 0.5 \notag
				\end{align}
				% \[
				% 	\vdots
				% \]
				\begin{align}
					\mu_{B_1}(u_{20}) &= \frac{\left(1-s_B(u_{20})\right)^2}{1 + \left(1-s_B(u_{20})\right)^2}\\
					&= \frac{\left(1-4\right)^2}{1 + \left(1-4\right)^2} \notag \\
					&= 0.90 \notag
				\end{align}
				% \[
				% 	\vdots
				% \]
				% \\
				% \\
				% \\
				\begin{align}
					\mu_{B_2}(u_1) &= \frac{\left(2-s_B(u_1)\right)^2}{1 + \left(2-s_B(u_1)\right)^2}\\
					&= \frac{\left(2-1\right)^2}{1 + \left(2-1\right)^2} \notag \\
					&= 0.5 \notag
				\end{align}
				\begin{align}
					\mu_{B_2}(u_2) &= \frac{\left(2-s_B(u_2)\right)^2}{1 + \left(2-s_B(u_2)\right)^2}\\
					&= \frac{\left(2-2\right)^5}{1 + \left(2-5\right)^2} \notag \\
					&= 0.9 \notag
				\end{align}
				\begin{align}
					\mu_{B_2}(u_3) &= \frac{\left(2-s_B(u_3)\right)^2}{1 + \left(2-s_B(u_3)\right)^2}\\
					&= \frac{\left(2-3\right)^2}{1 + \left(2-3\right)^2} \notag \\
					&= 0.5 \notag
				\end{align}
				\begin{align}
					\mu_{B_2}(u_4) &= \frac{\left(2-s_B(u_4)\right)^2}{1 + \left(2-s_B(u_4)\right)^2}\\
					&= \frac{\left(2-2\right)^2}{1 + \left(2-2\right)^2} \notag \\
					&= 0 \notag
				\end{align}
				\begin{align}
					\mu_{B_2}(u_5) &= \frac{\left(2-s_B(u_5)\right)^2}{1 + \left(2-s_B(u_5)\right)^2}\\
					&= \frac{\left(2-1\right)^2}{1 + \left(2-1\right)^2} \notag \\
					&= 0.5 \notag
				\end{align}
				\begin{align}
					\mu_{B_2}(u_6) &= \frac{\left(2-s_B(u_6)\right)^2}{1 + \left(2-s_B(u_6)\right)^2}\\
					&= \frac{\left(2-5\right)^2}{1 + \left(2-5\right)^2} \notag \\
					&= 0.9 \notag
				\end{align}
				\begin{align}
					\mu_{B_2}(u_7) &= \frac{\left(2-s_B(u_7)\right)^2}{1 + \left(2-s_B(u_7)\right)^2}\\
					&= \frac{\left(2-3\right)^2}{1 + \left(2-3\right)^2} \notag \\
					&= 0.5 \notag
				\end{align}
				\begin{align}
					\mu_{B_2}(u_8) &= \frac{\left(2-s_B(u_8)\right)^2}{1 + \left(2-s_B(u_8)\right)^2}\\
					&= \frac{\left(2-2\right)^2}{1 + \left(2-2\right)^2} \notag \\
					&= 0 \notag
				\end{align}
				\begin{align}
					\mu_{B_2}(u_9) &= \frac{\left(2-s_B(u_9)\right)^2}{1 + \left(2-s_B(u_9)\right)^2}\\
					&= \frac{\left(2-1\right)^2}{1 + \left(2-1\right)^2} \notag \\
					&= 0.5 \notag
				\end{align}
				\begin{align}
					\mu_{B_2}(u_{10}) &= \frac{\left(2-s_B(u_{10})\right)^2}{1 + \left(2-s_B(u_{10})\right)^2}\\
					&= \frac{\left(2-5\right)^2}{1 + \left(2-5\right)^2} \notag \\
					&= 0.9 \notag
				\end{align}
				\begin{align}
					\mu_{B_2}(u_{11}) &= \frac{\left(2-s_B(u_{11})\right)^2}{1 + \left(2-s_B(u_{11})\right)^2}\\
					&= \frac{\left(2-3\right)^2}{1 + \left(2-3\right)^2} \notag \\
					&= 0.5 \notag
				\end{align}
				\begin{align}
					\mu_{B_2}(u_{12}) &= \frac{\left(2-s_B(u_{12})\right)^2}{1 + \left(2-s_B(u_{12})\right)^2}\\
					&= \frac{\left(2-2\right)^2}{1 + \left(2-2\right)^2} \notag \\
					&= 0 \notag
				\end{align}
				\begin{align}
					\mu_{B_2}(u_{13}) &= \frac{\left(2-s_B(u_{13})\right)^2}{1 + \left(2-s_B(u_{13})\right)^2}\\
					&= \frac{\left(2-1\right)^2}{1 + \left(2-1\right)^2} \notag \\
					&= 0.5 \notag
				\end{align}
				\begin{align}
					\mu_{B_2}(u_{14}) &= \frac{\left(2-s_B(u_{14})\right)^2}{1 + \left(2-s_B(u_{14})\right)^2}\\
					&= \frac{\left(2-3\right)^2}{1 + \left(2-3\right)^2} \notag \\
					&= 0.5 \notag
				\end{align}
				\begin{align}
					\mu_{B_2}(u_{15}) &= \frac{\left(2-s_B(u_{15})\right)^2}{1 + \left(2-s_B(u_{15})\right)^2}\\
					&= \frac{\left(2-2\right)^2}{1 + \left(2-2\right)^2} \notag \\
					&= 0 \notag
				\end{align}
				\begin{align}
					\mu_{B_2}(u_{16}) &= \frac{\left(2-s_B(u_{16})\right)^2}{1 + \left(2-s_B(u_{16})\right)^2}\\
					&= \frac{\left(2-4\right)^2}{1 + \left(2-4\right)^2} \notag \\
					&= 0.8 \notag
				\end{align}
				\begin{align}
					\mu_{B_2}(u_{17}) &= \frac{\left(2-s_B(u_{17})\right)^2}{1 + \left(2-s_B(u_{17})\right)^2}\\
					&= \frac{\left(2-1\right)^2}{1 + \left(2-1\right)^2} \notag \\
					&= 0.5 \notag
				\end{align}
				\begin{align}
					\mu_{B_2}(u_{18}) &= \frac{\left(2-s_B(u_{18})\right)^2}{1 + \left(2-s_B(u_{18})\right)^2}\\
					&= \frac{\left(2-5\right)^2}{1 + \left(2-5\right)^2} \notag \\
					&= 0.9 \notag
				\end{align}
				\begin{align}
					\mu_{B_2}(u_{19}) &= \frac{\left(2-s_B(u_{19})\right)^2}{1 + \left(2-s_B(u_{19})\right)^2}\\
					&= \frac{\left(2-2\right)^2}{1 + \left(2-2\right)^2} \notag \\
					&= 0 \notag
				\end{align}
				% \[
				% 	\vdots
				% \]
				\begin{align}
					\mu_{B_2}(u_{20}) &= \frac{\left(2-s_B(u_{20})\right)^2}{1 + \left(2-s_B(u_{20})\right)^2}\\
					&= \frac{\left(2-4\right)^2}{1 + \left(2-4\right)^2} \notag \\
					&= 0.8 \notag
				\end{align}
				% \[
				% 	\vdots
				% \]
				% \\
				% \\
				% \\
				\begin{align}
					\mu_{B_3}(u_1) &= \frac{\left(3-s_B(u_1)\right)^2}{1 + \left(3-s_B(u_1)\right)^2}\\
					&= \frac{\left(3-1\right)^2}{1 + \left(3-1\right)^2} \notag \\
					&= 0.8 \notag
				\end{align}
				\begin{align}
					\mu_{B_3}(u_2) &= \frac{\left(3-s_B(u_2)\right)^2}{1 + \left(3-s_B(u_2)\right)^2}\\
					&= \frac{\left(3-2\right)^5}{1 + \left(3-5\right)^2} \notag \\
					&= 0.8 \notag
				\end{align}
				\begin{align}
					\mu_{B_3}(u_3) &= \frac{\left(3-s_B(u_3)\right)^2}{1 + \left(3-s_B(u_3)\right)^2}\\
					&= \frac{\left(3-3\right)^2}{1 + \left(3-3\right)^2} \notag \\
					&= 0 \notag
				\end{align}
				\begin{align}
					\mu_{B_3}(u_4) &= \frac{\left(3-s_B(u_4)\right)^2}{1 + \left(3-s_B(u_4)\right)^2}\\
					&= \frac{\left(3-2\right)^2}{1 + \left(3-2\right)^2} \notag \\
					&= 0.5 \notag
				\end{align}
				\begin{align}
					\mu_{B_3}(u_5) &= \frac{\left(3-s_B(u_5)\right)^2}{1 + \left(3-s_B(u_5)\right)^2}\\
					&= \frac{\left(3-1\right)^2}{1 + \left(3-1\right)^2} \notag \\
					&= 0.8 \notag
				\end{align}
				\begin{align}
					\mu_{B_3}(u_6) &= \frac{\left(3-s_B(u_6)\right)^2}{1 + \left(3-s_B(u_6)\right)^2}\\
					&= \frac{\left(3-5\right)^2}{1 + \left(3-5\right)^2} \notag \\
					&= 0.8 \notag
				\end{align}
				\begin{align}
					\mu_{B_3}(u_7) &= \frac{\left(3-s_B(u_7)\right)^2}{1 + \left(3-s_B(u_7)\right)^2}\\
					&= \frac{\left(3-3\right)^2}{1 + \left(3-3\right)^2} \notag \\
					&= 0 \notag
				\end{align}
				\begin{align}
					\mu_{B_3}(u_8) &= \frac{\left(3-s_B(u_8)\right)^2}{1 + \left(3-s_B(u_8)\right)^2}\\
					&= \frac{\left(3-2\right)^2}{1 + \left(3-2\right)^2} \notag \\
					&= 0.5 \notag
				\end{align}
				\begin{align}
					\mu_{B_3}(u_9) &= \frac{\left(3-s_B(u_9)\right)^2}{1 + \left(3-s_B(u_9)\right)^2}\\
					&= \frac{\left(3-1\right)^2}{1 + \left(3-1\right)^2} \notag \\
					&= 0.8 \notag
				\end{align}
				\begin{align}
					\mu_{B_3}(u_{10}) &= \frac{\left(3-s_B(u_{10})\right)^2}{1 + \left(3-s_B(u_{10})\right)^2}\\
					&= \frac{\left(3-5\right)^2}{1 + \left(3-5\right)^2} \notag \\
					&= 0.8 \notag
				\end{align}
				\begin{align}
					\mu_{B_3}(u_{11}) &= \frac{\left(3-s_B(u_{11})\right)^2}{1 + \left(3-s_B(u_{11})\right)^2}\\
					&= \frac{\left(3-3\right)^2}{1 + \left(3-3\right)^2} \notag \\
					&= 0 \notag
				\end{align}
				\begin{align}
					\mu_{B_3}(u_{12}) &= \frac{\left(3-s_B(u_{12})\right)^2}{1 + \left(3-s_B(u_{12})\right)^2}\\
					&= \frac{\left(3-2\right)^2}{1 + \left(3-2\right)^2} \notag \\
					&= 0.5 \notag
				\end{align}
				\begin{align}
					\mu_{B_3}(u_{13}) &= \frac{\left(3-s_B(u_{13})\right)^2}{1 + \left(3-s_B(u_{13})\right)^2}\\
					&= \frac{\left(3-1\right)^2}{1 + \left(3-1\right)^2} \notag \\
					&= 0.80 \notag
				\end{align}
				\begin{align}
					\mu_{B_3}(u_{14}) &= \frac{\left(3-s_B(u_{14})\right)^2}{1 + \left(3-s_B(u_{14})\right)^2}\\
					&= \frac{\left(3-3\right)^2}{1 + \left(3-3\right)^2} \notag \\
					&= 0 \notag
				\end{align}
				\begin{align}
					\mu_{B_3}(u_{15}) &= \frac{\left(3-s_B(u_{15})\right)^2}{1 + \left(3-s_B(u_{15})\right)^2}\\
					&= \frac{\left(3-2\right)^2}{1 + \left(3-2\right)^2} \notag \\
					&= 0.50 \notag
				\end{align}
				\begin{align}
					\mu_{B_3}(u_{16}) &= \frac{\left(3-s_B(u_{16})\right)^2}{1 + \left(3-s_B(u_{16})\right)^2}\\
					&= \frac{\left(3-4\right)^2}{1 + \left(3-4\right)^2} \notag \\
					&= 0.50 \notag
				\end{align}
				\begin{align}
					\mu_{B_3}(u_{17}) &= \frac{\left(3-s_B(u_{17})\right)^2}{1 + \left(3-s_B(u_{17})\right)^2}\\
					&= \frac{\left(3-1\right)^2}{1 + \left(3-1\right)^2} \notag \\
					&= 0.80 \notag
				\end{align}
				\begin{align}
					\mu_{B_3}(u_{18}) &= \frac{\left(3-s_B(u_{18})\right)^2}{1 + \left(3-s_B(u_{18})\right)^2}\\
					&= \frac{\left(3-5\right)^2}{1 + \left(3-5\right)^2} \notag \\
					&= 0.80 \notag
				\end{align}
				\begin{align}
					\mu_{B_3}(u_{19}) &= \frac{\left(3-s_B(u_{19})\right)^2}{1 + \left(3-s_B(u_{19})\right)^2}\\
					&= \frac{\left(3-2\right)^2}{1 + \left(3-2\right)^2} \notag \\
					&= 0.50 \notag
				\end{align}
				% \[
				% 	\vdots
				% \]
				\begin{align}
					\mu_{B_3}(u_{20}) &= \frac{\left(3-s_B(u_{20})\right)^2}{1 + \left(3-s_B(u_{20})\right)^2}\\
					&= \frac{\left(3-4\right)^2}{1 + \left(3-4\right)^2} \notag \\
					&= 0.50 \notag
				\end{align}
				\[
					\vdots
				\]
				% \\
				% \\
				% \\
				\begin{align}
					\mu_{B_4}(u_1) &= \frac{\left(4-s_B(u_1)\right)^2}{1 + \left(4-s_B(u_1)\right)^2}\\
					&= \frac{\left(4-1\right)^2}{1 + \left(4-1\right)^2} \notag \\
					&= 0.5 \notag
				\end{align}
				\begin{align}
					\mu_{B_4}(u_2) &= \frac{\left(4-s_B(u_2)\right)^2}{1 + \left(4-s_B(u_2)\right)^2}\\
					&= \frac{\left(4-2\right)^5}{1 + \left(4-5\right)^2} \notag \\
					&= 0.5 \notag
				\end{align}
				\begin{align}
					\mu_{B_4}(u_3) &= \frac{\left(4-s_B(u_3)\right)^2}{1 + \left(4-s_B(u_3)\right)^2}\\
					&= \frac{\left(4-3\right)^2}{1 + \left(4-3\right)^2} \notag \\
					&= 0.5 \notag
				\end{align}
				\begin{align}
					\mu_{B_4}(u_4) &= \frac{\left(4-s_B(u_4)\right)^2}{1 + \left(4-s_B(u_4)\right)^2}\\
					&= \frac{\left(4-2\right)^2}{1 + \left(4-2\right)^2} \notag \\
					&= 0.80 \notag
				\end{align}
				\begin{align}
					\mu_{B_4}(u_5) &= \frac{\left(4-s_B(u_5)\right)^2}{1 + \left(4-s_B(u_5)\right)^2}\\
					&= \frac{\left(4-1\right)^2}{1 + \left(4-1\right)^2} \notag \\
					&= 0.90 \notag
				\end{align}
				\begin{align}
					\mu_{B_4}(u_6) &= \frac{\left(4-s_B(u_6)\right)^2}{1 + \left(4-s_B(u_6)\right)^2}\\
					&= \frac{\left(4-5\right)^2}{1 + \left(4-5\right)^2} \notag \\
					&= 0.50 \notag
				\end{align}
				\begin{align}
					\mu_{B_4}(u_7) &= \frac{\left(4-s_B(u_7)\right)^2}{1 + \left(4-s_B(u_7)\right)^2}\\
					&= \frac{\left(4-3\right)^2}{1 + \left(4-3\right)^2} \notag \\
					&= 0.50 \notag
				\end{align}
				\begin{align}
					\mu_{B_4}(u_8) &= \frac{\left(4-s_B(u_8)\right)^2}{1 + \left(4-s_B(u_8)\right)^2}\\
					&= \frac{\left(4-2\right)^2}{1 + \left(4-2\right)^2} \notag \\
					&= 0.80 \notag
				\end{align}
				\begin{align}
					\mu_{B_4}(u_9) &= \frac{\left(4-s_B(u_9)\right)^2}{1 + \left(4-s_B(u_9)\right)^2}\\
					&= \frac{\left(4-1\right)^2}{1 + \left(4-1\right)^2} \notag \\
					&= 0.90 \notag
				\end{align}
				\begin{align}
					\mu_{B_4}(u_{10}) &= \frac{\left(4-s_B(u_{10})\right)^2}{1 + \left(4-s_B(u_{10})\right)^2}\\
					&= \frac{\left(4-5\right)^2}{1 + \left(4-5\right)^2} \notag \\
					&= 0.5 \notag
				\end{align}
				\begin{align}
					\mu_{B_4}(u_{11}) &= \frac{\left(4-s_B(u_{11})\right)^2}{1 + \left(4-s_B(u_{11})\right)^2}\\
					&= \frac{\left(4-3\right)^2}{1 + \left(4-3\right)^2} \notag \\
					&= 0.5 \notag
				\end{align}
				\begin{align}
					\mu_{B_4}(u_{12}) &= \frac{\left(4-s_B(u_{12})\right)^2}{1 + \left(4-s_B(u_{12})\right)^2}\\
					&= \frac{\left(4-2\right)^2}{1 + \left(4-2\right)^2} \notag \\
					&= 0.80 \notag
				\end{align}
				\begin{align}
					\mu_{B_4}(u_{13}) &= \frac{\left(4-s_B(u_{13})\right)^2}{1 + \left(4-s_B(u_{13})\right)^2}\\
					&= \frac{\left(4-1\right)^2}{1 + \left(4-1\right)^2} \notag \\
					&= 0.90 \notag
				\end{align}
				\begin{align}
					\mu_{B_4}(u_{14}) &= \frac{\left(4-s_B(u_{14})\right)^2}{1 + \left(4-s_B(u_{14})\right)^2}\\
					&= \frac{\left(4-3\right)^2}{1 + \left(4-3\right)^2} \notag \\
					&= 0.50 \notag
				\end{align}
				\begin{align}
					\mu_{B_4}(u_{15}) &= \frac{\left(4-s_B(u_{15})\right)^2}{1 + \left(4-s_B(u_{15})\right)^2}\\
					&= \frac{\left(4-2\right)^2}{1 + \left(4-2\right)^2} \notag \\
					&= 0.80 \notag
				\end{align}
				\begin{align}
					\mu_{B_4}(u_{16}) &= \frac{\left(4-s_B(u_{16})\right)^2}{1 + \left(4-s_B(u_{16})\right)^2}\\
					&= \frac{\left(4-4\right)^2}{1 + \left(4-4\right)^2} \notag \\
					&= 0 \notag
				\end{align}
				\begin{align}
					\mu_{B_4}(u_{17}) &= \frac{\left(4-s_B(u_{17})\right)^2}{1 + \left(4-s_B(u_{17})\right)^2}\\
					&= \frac{\left(4-1\right)^2}{1 + \left(4-1\right)^2} \notag \\
					&= 0.90 \notag
				\end{align}
				\begin{align}
					\mu_{B_4}(u_{18}) &= \frac{\left(4-s_B(u_{18})\right)^2}{1 + \left(4-s_B(u_{18})\right)^2}\\
					&= \frac{\left(4-5\right)^2}{1 + \left(4-5\right)^2} \notag \\
					&= 0.50 \notag
				\end{align}
				\begin{align}
					\mu_{B_4}(u_{19}) &= \frac{\left(4-s_B(u_{19})\right)^2}{1 + \left(4-s_B(u_{19})\right)^2}\\
					&= \frac{\left(4-2\right)^2}{1 + \left(4-2\right)^2} \notag \\
					&= 0.80 \notag
				\end{align}
				% \[
				% 	\vdots
				% \]
				\begin{align}
					\mu_{B_4}(u_{20}) &= \frac{\left(4-s_B(u_{20})\right)^2}{1 + \left(4-s_B(u_{20})\right)^2}\\
					&= \frac{\left(4-4\right)^2}{1 + \left(4-4\right)^2} \notag \\
					&= 0 \notag
				\end{align}
				\[
					\vdots
				\]
				% \\
				% \\
				% \\
				\begin{align}
					\mu_{B_5}(u_1) &= \frac{\left(5-s_B(u_1)\right)^2}{1 + \left(5-s_B(u_1)\right)^2}\\
					&= \frac{\left(5-1\right)^2}{1 + \left(5-1\right)^2} \notag \\
					&= 0.94 \notag
				\end{align}
				\begin{align}
					\mu_{B_5}(u_2) &= \frac{\left(5-s_B(u_2)\right)^2}{1 + \left(5-s_B(u_2)\right)^2}\\
					&= \frac{\left(5-2\right)^5}{1 + \left(5-5\right)^2} \notag \\
					&= 0 \notag
				\end{align}
				\begin{align}
					\mu_{B_5}(u_3) &= \frac{\left(5-s_B(u_3)\right)^2}{1 + \left(5-s_B(u_3)\right)^2}\\
					&= \frac{\left(5-3\right)^2}{1 + \left(5-3\right)^2} \notag \\
					&= 0.80 \notag
				\end{align}
				\begin{align}
					\mu_{B_5}(u_4) &= \frac{\left(5-s_B(u_4)\right)^2}{1 + \left(5-s_B(u_4)\right)^2}\\
					&= \frac{\left(5-2\right)^2}{1 + \left(5-2\right)^2} \notag \\
					&= 0.90 \notag
				\end{align}
				\begin{align}
					\mu_{B_5}(u_5) &= \frac{\left(5-s_B(u_5)\right)^2}{1 + \left(5-s_B(u_5)\right)^2}\\
					&= \frac{\left(5-1\right)^2}{1 + \left(5-1\right)^2} \notag \\
					&= 0.94 \notag
				\end{align}
				\begin{align}
					\mu_{B_5}(u_6) &= \frac{\left(5-s_B(u_6)\right)^2}{1 + \left(5-s_B(u_6)\right)^2}\\
					&= \frac{\left(5-5\right)^2}{1 + \left(5-5\right)^2} \notag \\
					&= 0 \notag
				\end{align}
				\begin{align}
					\mu_{B_5}(u_7) &= \frac{\left(5-s_B(u_7)\right)^2}{1 + \left(5-s_B(u_7)\right)^2}\\
					&= \frac{\left(5-3\right)^2}{1 + \left(5-3\right)^2} \notag \\
					&= 0.80 \notag
				\end{align}
				\begin{align}
					\mu_{B_5}(u_8) &= \frac{\left(5-s_B(u_8)\right)^2}{1 + \left(5-s_B(u_8)\right)^2}\\
					&= \frac{\left(5-2\right)^2}{1 + \left(5-2\right)^2} \notag \\
					&= 0.90 \notag
				\end{align}
				\begin{align}
					\mu_{B_5}(u_9) &= \frac{\left(5-s_B(u_9)\right)^2}{1 + \left(5-s_B(u_9)\right)^2}\\
					&= \frac{\left(5-1\right)^2}{1 + \left(5-1\right)^2} \notag \\
					&= 0.94 \notag
				\end{align}
				\begin{align}
					\mu_{B_5}(u_{10}) &= \frac{\left(5-s_B(u_{10})\right)^2}{1 + \left(5-s_B(u_{10})\right)^2}\\
					&= \frac{\left(5-5\right)^2}{1 + \left(5-5\right)^2} \notag \\
					&= 0 \notag
				\end{align}
				\begin{align}
					\mu_{B_5}(u_{11}) &= \frac{\left(5-s_B(u_{11})\right)^2}{1 + \left(5-s_B(u_{11})\right)^2}\\
					&= \frac{\left(5-3\right)^2}{1 + \left(5-3\right)^2} \notag \\
					&= 0.80 \notag
				\end{align}
				\begin{align}
					\mu_{B_5}(u_{12}) &= \frac{\left(5-s_B(u_{12})\right)^2}{1 + \left(5-s_B(u_{12})\right)^2}\\
					&= \frac{\left(5-2\right)^2}{1 + \left(5-2\right)^2} \notag \\
					&= 0.90 \notag
				\end{align}
				\begin{align}
					\mu_{B_5}(u_{13}) &= \frac{\left(5-s_B(u_{13})\right)^2}{1 + \left(5-s_B(u_{13})\right)^2}\\
					&= \frac{\left(5-1\right)^2}{1 + \left(5-1\right)^2} \notag \\
					&= 0.94 \notag
				\end{align}
				\begin{align}
					\mu_{B_5}(u_{14}) &= \frac{\left(5-s_B(u_{14})\right)^2}{1 + \left(5-s_B(u_{14})\right)^2}\\
					&= \frac{\left(5-3\right)^2}{1 + \left(5-3\right)^2} \notag \\
					&= 0.80 \notag
				\end{align}
				\begin{align}
					\mu_{B_5}(u_{15}) &= \frac{\left(5-s_B(u_{15})\right)^2}{1 + \left(5-s_B(u_{15})\right)^2}\\
					&= \frac{\left(5-2\right)^2}{1 + \left(5-2\right)^2} \notag \\
					&= 0.90 \notag
				\end{align}
				\begin{align}
					\mu_{B_5}(u_{16}) &= \frac{\left(5-s_B(u_{16})\right)^2}{1 + \left(5-s_B(u_{16})\right)^2}\\
					&= \frac{\left(5-4\right)^2}{1 + \left(5-4\right)^2} \notag \\
					&= 0.50 \notag
				\end{align}
				\begin{align}
					\mu_{B_5}(u_{17}) &= \frac{\left(5-s_B(u_{17})\right)^2}{1 + \left(5-s_B(u_{17})\right)^2}\\
					&= \frac{\left(5-1\right)^2}{1 + \left(5-1\right)^2} \notag \\
					&= 0.94 \notag
				\end{align}
				\begin{align}
					\mu_{B_5}(u_{18}) &= \frac{\left(5-s_B(u_{18})\right)^2}{1 + \left(5-s_B(u_{18})\right)^2}\\
					&= \frac{\left(5-5\right)^2}{1 + \left(5-5\right)^2} \notag \\
					&= 0 \notag
				\end{align}
				\begin{align}
					\mu_{B_5}(u_{19}) &= \frac{\left(5-s_B(u_{19})\right)^2}{1 + \left(5-s_B(u_{19})\right)^2}\\
					&= \frac{\left(5-2\right)^2}{1 + \left(5-2\right)^2} \notag \\
					&= 0.90 \notag
				\end{align}
				% \[
				% 	\vdots
				% \]
				\begin{align}
					\mu_{B_5}(u_{20}) &= \frac{\left(5-s_B(u_{20})\right)^2}{1 + \left(5-s_B(u_{20})\right)^2}\\
					&= \frac{\left(5-4\right)^2}{1 + \left(5-4\right)^2} \notag \\
					&= 0.50 \notag
				\end{align}
				% \\
				% \\
				% \\
			\paragraph{Now,}
				Since our sample problem set defines two Constraints, $\mu_{A_i}(u_j)$ and $\mu_{B_i}(u_j) \textit{ for i=1,2,...,5 and j=1,2,...,20}$, \\
				We will now proceed to estimate a single membership probability $\mu_i(u_j)  \textit{ for i=1,2,...,5 and j=1,2,...,20}$, which is estimated by the following:
				\begin{equation}
					\mu_i(u_j) = \mu_{A_i}(u_j) \cdot \mu_{B_i}(u_j)
				\end{equation}
				\begin{align}
					\mu_1(u_1) &= \mu_{A_1}(u_1) \cdot \mu_{B_1}(u_1) \\
					&= 0.50 \times 0 \notag \\
					&= 0 \notag\\
					\mu_1(u_2) &= \mu_{A_1}(u_2) \cdot \mu_{B_1}(u_2) \\
					&= 0.50 \times 0.94 \notag \\
					&= 0.47 \notag\\
					\mu_1(u_3) &= \mu_{A_1}(u_3) \cdot \mu_{B_1}(u_3) \\
					&= 0.50 \times 0.50 \notag \\
					&= 0.25 \notag\\
					&\vdots \notag \\
					\mu_3(u_4) &= \mu_{A_3}(u_4) \cdot \mu_{B_3}(u_4) \\
					&= 0.50 \times 0.80 \notag \\
					&= 0.40 \notag\\
					&\vdots \notag \\
					\mu_5(u_{20}) &= \mu_{A_5}(u_{20}) \cdot \mu_{B_5}(u_{20}) \\
					&= 0.80 \times 0.50 \notag \\
					&= 0.40 \notag\\
				\end{align}
				Which we can more concisely represent in the following sets:
				\begin{align}
					\mu_i = \{ \mu_i(u_1), \mu_i(u_2), \mu_i(u_3),\dots,\mu_i(u_20) \} \qquad \textit{for i=1,2,\dots,5}
				\end{align}
				So that
				\begin{align}
					\mu_1 &= \{ \mu_1(u_1), \mu_1(u_2), \mu_1(u_3),\dots,\mu_1(u_20) \} \\
					&= \{ 0, 0.47, 0.40,\dots,0.90 \} \notag \\
					\mu_2 &= \{ \mu_2(u_1), \mu_2(u_2), \mu_2(u_3),\dots,\mu_2(u_20) \}\\
					&= \{ 0, 0, 0,\dots,0.40 \} \notag \\
					\mu_3 &= \{ \mu_3(u_1), \mu_3(u_2), \mu_3(u_3),\dots,\mu_3(u_20) \}\\
					&= \{ 0.40, 0.40, 0,\dots,0 \} \notag \\
					\mu_4 &= \{ \mu_4(u_1), \mu_4(u_2), \mu_4(u_3),\dots,\mu_4(u_20) \}\\
					&= \{ 0.40, 0.40, 0.40,\dots,0 \} \notag \\
					\mu_5 &= \{ \mu_5(u_1), \mu_5(u_2), \mu_5(u_3),\dots,\mu_5(u_20) \}\\
					&= \{ 0.84, 0, 0.72,\dots,0.40 \} \notag					
				\end{align}
				Summarized fully in the table below:
				
				\begin{table}[!th]
					\begin{center}
						\begin{tabular}{|l|c|c|c|c|c|c|}
							\hline
							\textbf{$i$} & \textbf{$\mu_{A_1}(u_i)$} & \textbf{$\mu_{B_1}(u_i)$} & \textbf{$\mu_{A_1}\cdot\mu_{B_1}$} & \textbf{$\mu_{A_2}(u_i)$} & \textbf{$\mu_{B_2}(u_i)$} & \textbf{$\mu_{A_2}(u_i)\cdot\mu_{B_2}(u_i)$}\\
							\hline
							1. & 0.5 & 0 & 0 & 0 & 0.5 & 0 \\
							\hline
							2. & 0.5 & 0.94 & 0.47 & 0 & 0.5 & 0 \\
							\hline
							3. & 0.5 & 0.80 & 0.40 & 0 & 0.5 & 0 \\
							\hline
							4. & 0.5 & 0.5 & 0.25 & 0 & 0 & 0 \\
							\hline
							5. & 0.9 & 0 & 0 & 0.8 & 0.5 & 0.4 \\
							\hline
							6. & 0.9 & 0.94 & 0.85 & 0.8 & 0.9 & 0.72 \\
							\hline
							7. & 0.9 & 0.8 & 0.72 & 0.8 & 0.5 & 0.4 \\
							\hline
							8. & 0.9 & 0.5 & 0.45 & 0.8 & 0 & 0 \\
							\hline
							9. & 0 & 0 & 0 & 0 & 0.5 & 0 \\
							\hline
							10. & 0 & 0 & 0 & 0.5 & 0.5 & 0.25 \\
							\hline
							11. & 0 & 0.8 & 0 & 0.5 & 0.5 & 0.25 \\
							\hline
							12. & 0 & 0.5 & 0 & 0.5 & 0.5 & 0.25 \\
							\hline
							13. & 0.94 & 0 & 0 & 0.9 & 0.5 & 0.45 \\
							\hline
							14. & 0.94 & 0 & 0 & 0 & 0.5 & 0 \\
							\hline
							15. & 0.94 & 0.8 & 0.75 & 0.7 & 0.5 & 0.45 \\
							\hline
							16. & 0.94 & 0.9 & 0.85 & 0.9 & 0.8 & 0.72 \\
							\hline
							17. & 0.8 & 0 & 0 & 0.5 & 0.5 & 0.25 \\
							\hline
							18. & 0.8 & 0.94 & 0.75 & 0.5 & 0.9 & 0.45 \\
							\hline
							19. & 0.8 & 0.5 & 0.4 & 0.5 & 0 & 0 \\
							\hline
							20. & 0.8 & 0.9 & 0.72 & 0.5 & 0.8 & 0.40 \\
							\hline
						\end{tabular}
						\caption{The Membership Degree Distribution for the Sample Assignment Problem, I of III}
						\label{ex:table2}
					\end{center}
				\end{table}
				\begin{table}[!th]
					\begin{center}
						\begin{tabular}{|l|c|c|c|c|c|c|}
							\hline
							\textbf{$i$} & \textbf{$\mu_{A_3}(u_i)$} & \textbf{$\mu_{B_3}(u_i)$} & \textbf{$\mu_{A_3}(u_i)\cdot\mu_{B_3}(u_i)$} & \textbf{$\mu_{A_4}(u_i)$} & \textbf{$\mu_{B_4}(u_i)$} & \textbf{$\mu_{A_4}(u_i)\cdot\mu_{B_4}(u_i)$}\\
							\hline
							1. & 0.5 & 0.8 & 0.4 & 0.8 & 0.5 & 0.4 \\
							\hline
							2. & 0.5 & 0.8 & 0.4 & 0.8 & 0.5 & 0.4 \\
							\hline
							3. & 0.5 & 0 & 0 & 0.8 & 0.5 & 0.4 \\
							\hline
							4. & 0.5 & 0.5 & 0.25 & 0.8 & 0.8 & 0.64 \\
							\hline
							5. & 0.5 & 0.8 & 0.4 & 0 & 0.9 & 0 \\
							\hline
							6. & 0.5 & 0.8 & 0.4 & 0 & 0.5 & 0 \\
							\hline
							7. & 0.5 & 0 & 0 & 0 & 0.5 & 0 \\
							\hline
							8. & 0.5 & 0.5 & 0.25 & 0 & 0.8 & 0 \\
							\hline
							9. & 0.8 & 0.8 & 0.64 & 0.9 & 0.9 & 0.81 \\
							\hline
							10. & 0.8 & 0.8 & 0.64 & 0.9 & 0.5 & 0.45 \\
							\hline
							11. & 0.8 & 0 & 0 & 0 & 0.5 & 0.45 \\
							\hline
							12. & 0.8 & 0.5 & 0.45 & 0.9 & 0.8 & 0.72 \\
							\hline
							13. & 0.8 & 0.8 & 0.64 & 0.5 & 0.9 & 0.45 \\
							\hline
							14. & 0.8 & 0 & 0 & 0.5 & 0.5 & 0.25 \\
							\hline
							15. & 0.8 & 0.5 & 0.4 & 0.5 & 0.8 & 0.4 \\
							\hline
							16. & 0.8 & 0.5 & 0.4 & 0.5 & 0 & 0 \\
							\hline
							17. & 0.9 & 0.8 & 0 & 0.5 & 0.9 & 0.45 \\
							\hline
							18. & 0 & 0.8 & 0 & 0.5 & 0.5 & 0.25 \\
							\hline
							19. & 0 & 0.5 & 0 & 0.5 & 0.8 & 0.4 \\
							\hline
							20. & 0 & 0.5 & 0 & 0.5 & 0 & 0 \\
							\hline
						\end{tabular}
						\caption{The Membership Degree Distribution for the Sample Assignment Problem, II of III}
						\label{ex:table3}
					\end{center}					
				\end{table}
				\begin{table}[!th]
					\begin{center}
						\begin{tabular}{|l|c|c|c|c|c|c|}
							\hline
							\textbf{$i$} & \textbf{$\mu_{A_5}(u_i)$} & \textbf{$\mu_{B_5}(u_i)$} & \textbf{$\mu_{A_5}(u_i)\cdot\mu_{B_5}(u_i)$}\\
							\hline
							1. & 0.9 & 0.94 & 0.84 \\
							\hline
							2. & 0.9 & 0 & 0 \\
							\hline
							3. & 0.9 & 0.8 & 0.72 \\
							\hline
							4. & 0.9 & 0.9 & 0.81 \\
							\hline
							5. & 0.5 & 0.94 & 0.47 \\
							\hline
							6. & 0.5 & 0 & 0 \\
							\hline
							7. & 0.5 & 0.8 & 0.4 \\
							\hline
							8. & 0.5 & 0.9 & 0.45 \\
							\hline
							9. & 0.94 & 0.94 & 0.88 \\
							\hline
							10. & 0.94 & 0 & 0 \\
							\hline
							11. & 0.94 & 0.8 & 0.75 \\
							\hline
							12. & 0.94 & 0.9 & 0.84 \\
							\hline
							13. & 0 & 0.94 & 0 \\
							\hline
							14. & 0 & 0.8 & 0 \\
							\hline
							15. & 0 & 0.90 & 0 \\
							\hline
							16. & 0 & 0.5 & 0 \\
							\hline
							17. & 0.8 & 0.94 & 0.75 \\
							\hline
							18. & 0.8 & 0 & 0 \\
							\hline
							19. & 0.8 & 0.9 & 0.72 \\
							\hline
							20. & 0.8 & 0.5 & 0.4 \\
							\hline
						\end{tabular}
						\caption{The Membership Degree Distribution for the Sample Assignment Problem, III of III}
						\label{ex:table4}
					\end{center}					
				\end{table}
		\section{Populating the States(Partitions)}
			\paragraph{}
				Having achieved a well defined membership probability for each of the graduates into each of the states, in the previous section, we now proceed to assigning the most qualifying candidates to each state, while not exceeding each State's defined expected partition size, capacity that is.
				Recall that from eq(4.9) to eq(4.18), we defined the expected assignment size of each state to be:
				\begin{align}
					\mid p_1 \mid &= 3 \notag \\
					\mid p_2 \mid &= 2 \notag \\
					\mid p_3 \mid &= 1 \notag \\
					\mid p_4 \mid &= 5 \notag \\
					\mid p_5 \mid &= 9 \notag
				\end{align}
				Now, for each State,$p_i$, we select the first $\mid p_i \mid$ candidates, $u_j$ with the highest membership $\mu_i(u_j)$ probability in order, for each State i=1,2,...,5:
				\begin{align}
					p_1 &= \{u_6(0.85), u_{16}(0.84), u_{14}(0.74)\}\\
					p_2 &= \{u_{10}(0.45), u_{13}(0.45)\}\\
					p_3 &= \{u_{9}(0.64)\}\\
					p_4 &= \{u_{12}(0.72),u_{4}(0.64),u_{11}(0.45),u_{17}(0.45),u_{1}(0.45)\}\\
					p_5 &= \{u_{3}(0.72),u_{19}(0.72),u_{8}(0.45),u_{5}(0.47),u_{7}(0.40),u_{20}(0.40),u_{2}(0),u_{18}(0),u_{15}(0)\}
				\end{align}
				After successfully applying the algorithm developed in the previous section to make the assignment on the sample problem, we have been able to reach an assignment explained thus:
				\subsection{The Assignment Table}
					\paragraph{}
						\begin{table}[!th]
							\begin{center}
								\begin{tabular}{|l|c|c|c|c|}
									\hline
									\textbf{Name} & \textbf{State of Origin} & \textbf{College} & \textbf{Posting} & \textbf{Membership Degree} \\
									\hline
									GR06 & Akwa-Ibom & BUK, Kano & Osun & 85\% \\
									\hline
									GR16 & Kano & AKSU, Akwa-Ibom & Osun & 84\% \\
									\hline
									GR14 & Kano & NSUKKA, Enugu & Osun & 74\% \\
									\hline
									GR10 & Osun & BUK, Kano & Kwara & 45\% \\
									\hline
									GR13 & Kano & OUI, Osun & Kwara & 45\% \\
									\hline
									GR09 & Osun & OAU, Osun & Enugu & 64\% \\
									\hline
									GR12 & Osun & UNILORIN, Kwara & Akwa-Ibom & 72\% \\
									\hline
									GR04 & Kwara & UNILORIN, Kwara & Akwa-Ibom & 64\% \\
									\hline
									GR11 & Osun & NSUKKA, Enugu & Akwa-Ibom & 45\% \\
									\hline
									GR17 & Enugu & UNIOSUN, Osun & Akwa-Ibom & 45\% \\
									\hline
									GR01 & Kwara & OAU, Osun & Akwa-Ibom & 45\% \\
									\hline
									GR03 & Kwara & NSUKKA, Enugu & Kano & 88\% \\
									\hline
									GR19 & Enugu & UNILORIN, Kwara & Kano & 72\% \\
									\hline
									GR08 & Akwa-Ibom & UNILORIN, Kwara & Kano & 72\% \\
									\hline
									GR05 & Akwa-Ibom & OAU, Osun & Kano & 47\% \\
									\hline
									GR07 & Akwa-Ibom & NSUKKA, Enugu & Kano & 45\% \\
									\hline
									GR20 & Enugu & AKSU, Akwa-Ibom & Kano & 40\% \\
									\hline
									GR02 & Kwara & BUK, Kano & Kano & 0\% \\
									\hline
									GR18 & Enugu & BUK, Kano & Kano & 0\% \\
									\hline
									GR15 & Kano & UNILORIN, Kwara & Kano & 0\% \\
									\hline
								\end{tabular}
								\caption{Showing the NYSC Deployment Schedule}
								\label{ex:table}
							\end{center}							
						\end{table}
	\chapter{Discussion and Conclusion}
		\section{Discussion of Results}
			\paragraph{}
				From the assignment achieved in the previous Chapter, the total optimal assignment is 17 out of 20, which is a score of 85\%, with an average membership score of 51.6\%. This algorithm can be assessed to have performed good on the task, nonethless, we point several points of possible improvements that can be explored to enhance the result.
		\section{Improvement Suggestions}
			Although, the sample problem outlined in the previous chapter got a 90\% success assignment, by making a few adjustments, we observe that the accuracy can be raised as high as 100\%
			\begin{itemize}
				\item When choosing objects into each partition, keep a tab on the number of non-zero assignable members left for each class, if this number is equal to the expected partition size of any class, proceed to make assignments into the class in question. This precaution ensures that each class gets the opportunity to get the maximum number assignments with fitness value different from zero.
			\end{itemize}
		\section{Further Work}
			As must have been observed in the algorithm designed in this text, an attempt was made to solve a simplified version of the University Timetabling problem. Here, the Constraints are of a simpler construct with no inter-constraint interaction. Such that, by simpley multiiplying out the individual constraints, we were able to estimate the degree of membership of each object in each partition. In the case of the University Timetabling problem, several constraints have more complex interactions which will be further studied.\\
			Thank you.	
	\newpage
	\bibliography{ch1}
	\bibliographystyle{apalike}
	\addcontentsline{toc}{chapter}{Bibliography}
\end{document}